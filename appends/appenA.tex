%!TEX root = ../PTC-LibUG.tex

\cleardoublepage
\chapter{Internal States}
\label{app:states}

\index{internal states!described}
\index{global variable!\ptc{default}}
\index{default@\ptc{default}!global variable}
\index{flags!internal state}
\index{flag!\ptc{totalpath}}
\index{totalpath@\ptc{totalpath}!flag}
\index{flag!\ptc{time}}
\index{time@\ptc{time}!flag}
\index{radiation!internal state}
\index{flag!\ptc{radiation}}
\index{radiation@\ptc{radiation}!flag}
\index{RF cavity!internal state}
\index{flag!\ptc{nocavity}}
\index{nocavity@\ptc{nocavity}!flag}
\index{fringe field!internal state}
\index{flag!\ptc{fringe}}
\index{fringe@\ptc{fringe}!flag}
\index{misalignment!internal state}
\index{flag!\ptc{exactmis}}
\index{exactmis@\ptc{exactmis}!flag}
%
\PTC\ contains a series of flags held in the global variable \ptc{DEFAULT}.
These \emph{internal-state} flags allow you to control certain aspects of \PTC's behavior
based on the needs of your simulation. For example, you could run one simulation
with the \ptc{RADIATION }flag off and another simulation with the flag on.
By comparing the results, you would answer the question: ``How important is
radiation to my simulation?''

Here are six internal-state flags and the \PTC\ behavior each flag controls:

\begin{enumerate}
  \item TOTALPATH ensures a computation of the total path length
or total time of flight.
  \item TIME selects time of flight rather than path length (\emph{cT} to
be precise).
  \item RADIATION turns on classical radiation.
  \item NOCAVITY forces the code to ignore RF cavities. It has also implications
on the normal form if performed in three degrees of freedom.
  \item FRINGE turns on quadrupole fringe fields based on the \emph{b$_{2}$}
and \emph{a$_{2}$} components in the element.
  \item EXACTMIS forces the misalignments to be treated exactly.
\end{enumerate}

\PTC\ has a special internal state called \ptc{DEFAULT }that selects path length
rather than time of flight, does not ensure a computation of the total path length,
turns off classical radiation, includes RF cavities, turns off quadrupole fringe fields,
and does not treat misalignments exactly.

\index{routine!\ptc{MAKE_STATES}}
\index{MAKE_STATES@\ptc{MAKE_STATES}!routine}
\index{routine!\ptc{UPDATE_STATES}}
\index{UPDATE_STATES@\ptc{UPDATE_STATES}!routine}
This example shows how to set the default internal-state environment described
above:

\begin{ptccode}
CALL MAKE_STATES(.FALSE.)\\
EXACT_MODEL=.TRUE.\\
DEFAULT=DEFAULT\\
CALL UPDATE_STATES\\
\end{ptccode}

\ptc{MAKE_STATES }is \ptc{TRUE }for electrons and \ptc{FALSE }for protons.
When \ptc{EXACT_MODEL }is  \ptc{TRUE}, \PTC\ uses the full square-root Hamiltonian.

The next example shows how to set the default internal-state environment with the
NOCAVITY and EXACTMIS flags:

\begin{ptccode}
CALL MAKE_STATES(.FALSE.)\\
EXACT_MODEL=.TRUE.\\
DEFAULT=DEFAULT+NOCAVITY+EXACTMIS\\
CALL UPDATE_STATES\\
\end{ptccode}

\index{knob!flag}
\index{flag!\ptc{para_in}}
\index{para_in@\ptc{para_in}!flag}
\index{phase space!flag}
\index{normal form!phase-space dimensions}
\index{flag!\ptc{only_4d}}
\index{only_4d@\ptc{only_4d}!flag}
\index{energy!flag}
\index{flag!\ptc{delta}}
\index{delta@\ptc{delta}!flag}
\index{spin!flag}
\index{flag!\ptc{spin}}
\index{spin@\ptc{spin}!flag}
\index{flag!\ptc{spin_only}}
\index{spin_only@\ptc{spin_only}!flag}
\index{flag!\ptc{spin_dim}}
\index{spin_dim@\ptc{spin_dim}!flag}
The following six flags are strictly related to TPSA calculations:

\begin{enumerate}
  \item If PARA_IN is specified, TPSA knobs are included in the calculation. It
is activated by a unary + on a state, for example, \ptc{TRACK(PSR,Y,1,+DEFAULT)}.
  \item If ONLY_4D is specified, then neither path length nor time is a TPSA variable.
This means that the phase-space dimension in the normal form will
be 4 (\emph{X}, \emph{P}$_{x}$, \emph{Y}, \emph{P}$_{y}$). Also X(5) will not be
TPSA unless DELTA is also specified (see next the item). This flag has no effect on
\PTC\ tracking. \PTC\ always tracks six phase-space variables.
  \item If DELTA is specified, then ONLY_4D is also true. However, in this case,
X(5) is the fifth TPSA variable:  \emph{X}, \emph{P}$_{x}$, \emph{Y}, \emph{P}$_{y}$ plus
Energy as the fifth variable. The phase-space dimension in the normal
form will also be 4; momentum compaction cannot be computed.
  \item SPIN
  \item SPIN_ONLY
  \item SPIN_DIM
\end{enumerate}

This example shows how to set the default internal-state environment with a
four-dimensional phase space in the normal form:

\begin{ptccode}
DEFAULT=DEFAULT+ONLY_4D\\
\end{ptccode}

The next example shows how to set the default internal-state environment with a
four-dimensional phase space in the normal form and with energy as the fifth-dimension
TPSA variable:

\begin{ptccode}
DEFAULT=DEFAULT+DELTA\\
\end{ptccode}

\endinput
