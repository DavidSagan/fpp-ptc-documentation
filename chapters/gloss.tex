%!TEX root = ../PTC-LibUG.tex

%\cleardoublepage
\chapter{Glossary}

\begin{description}
  \item[accelerator topology]
A description in spatial terms of the magnets in an accelerator and
how they interact with particles.

  \item[affine space]
An abstract space that generalizes Euclidean space's affine-geometric properties.
In affine space, unlike Euclidean space, no distinguished point serves as an origin.
\fxnote{Al: I reworded the Wikipedia version of this definition and
added ``affine transformation'' below. My definitions need to be reviewed for accuracy.}

  \item[affine transformation]
A transformation between two affine spaces that consists of a linear transformation
followed by a translation.

  \item[anharmonicity]
``The deviation of a system from being a harmonic oscillator. An
oscillator that is not oscillating in simple harmonic motion is known
as an anharmonic oscillator where the system can be approximated to
a harmonic oscillator and the anharmonicity can be calculated using
perturbation theory. If the anharmonicity is large then other numerical
techniques have to be used.'' From Wikipedia. Anharmonicity is a
global property of an accelerator.

  \item[atlas]
A collection of charts.

  \item[beam line]
1) In a particle accelerator, the line that a beam of particles travels
through an ordered sequence of elements. 2) In \PTC, a \emph{layout
}\textendash{} a linked list of containers called fibres. 3) In \PTC,
a \emph{node layout }\textendash{} an \emph{expanded }beam line \textendash{}
a linked list of integration nodes.

  \item[beta function]
The scale of particle motion at a point. The beta function is a local property of an accelerator.

  \item[cavity]
In \PTC, either a standard pill-box cavity or a traveling wave cavity.

  \item[{\textls[50]{CEBAF}}]
See \emph{Continuous Electron Beam Accelerator Facility.}

  \item[chart]
A data type containing a frame of reference that locates a fibre (and the beamline element
the fibre points to) within the lattice's three-dimensional space.

  \item[chromaticity]
A parameter that describes how the tunes change with particle energy. Chromaticity
is a global property of an accelerator.

  \item[closed orbit]
An orbit of a particle around a recirculating accelerator that begins
and ends at exactly the same position and momentum.

  \item[Continuous Electron Beam Accelerator Facility]
A recirculating linear electron accelerator operated by Thomas Jefferson
National Laboratory (JLab).

  \item[dispersion]
To be provided. Local property.

  \item[{\textls[50]{DNA}}]
A database of \emph{\textls[50]{DNA} sequences}.
The database contains all the beamline elements in the lattice.

  \item[DNA sequence]
A layout in which all the elements appear only once. A DNA sequence
is not used for tracking; it provides a database.
See \textls[50]{\emph{DNA}}.

  \item[dynamical Euclidean group]
An operation that combines elements in a group.

  \item[element]
1) A component of a beam line, for example, a magnet, a drift, or
an RF cavity. 2) In \PTC, a data type representing a beamline
element.

  \item[{\textls[50]{FFAG}}]
See \emph{fixed-field alternating gradient synchrotron.}

  \item[Fixed-field alternating gradient synchrotron]
``An alternative technology for accelerating charged particles,
including protons. It could provide a cheaper and more compact alternative
to existing accelerator and synchrotron technologies.''
From: \url{http://www.neutrons.cclrc.ac.uk/Report/glossary.aspx}.

  \item[{\textls[50]{FPP}}]
See \emph{Fully Polymorphic Package}.

  \item[fibre]
A data type containing a pointer to a beamline element as well as pointers
to the fibres that precede and follow it in the particle path.
When linked together, fibres follow the particle trajectory and define
the beam line. Multiple fibres can point to the same beamline element.  A
fibre also contains a pointer to a \emph{chart} that locates the element within
the global frame of reference for the lattice, pointers to sets of \emph{patches}
that connect the elements of successive fibres geometrically, energetically,
and temporally, and the direction of propagation through the element.

  \item[fringe field]
The magnetic field at each end of a magnet, where the magnetic force
stops being constant and falls to zero.

  \item[Fully Polymorphic Package]
A package of polymorphic types and tools to extract a Poincar\'e map
from \PTC\ (or another symplectic integrator) and analyze the map. The
most common analysis tool is the normal form. \FPP\ creates a Taylor-Real
polymorphic type which changes shape at execution time. \FPP\ replaces
the real variable ``real(8)'' with a new type called REAL 8 to
produce Taylor series for analysis.

  \item[girder]
A collection of siamese and regular elements that can be moved as
a group.

  \item[integration node]
A data type that tracks particles as they move through a fibre. The
integration node is the basic unit of \PTC\ tracking. Each fibre points to
N+4 integration nodes: entrance patch (and misalignment, if any),
entrance fringe field, body of element (which has any number N of
integration nodes), exit fringe field, and exit patch.

  \item[integration step]
A step that tracks a particle or a polymorph as it moves through an
integration node in the body of an element. If the body of an element
has one integration node, \PTC\ performs one integration step. If the body
of an element has been split into four integration nodes, \PTC\ performs four
integration steps. The integration nodes for the entrance patch, entrance
fringe field, exit fringe field, and exit patch do not require integration steps.
Depending on the integration method, \PTC\ may apply one kick or
multiple kicks per integration step.

  \item[kicker]
A fictitious magnet used to make magnets line up. \PTC\ does not use
kickers.

   \item[knob]
A polymorph that turns itself into a simple Taylor series when used. Knobs
let users set up parameters that can be changed without having to recompile.

  \item[Large Hadron Collider]
A particle accelerator and hadron collider with two intersecting rings,
currently under construction at CERN.

  \item[layout]
A data type that represents a beam line as a doubly-linked list of
fibres. The layout follows the particle path by specifying the order
of the fibres, which point to an actual element in a beam line.

  \item[LEGO\ block]
An interlocking plastic block or brick in sets of building toys manufactured
by the LEGO Group, a company in Denmark.  Beamline elements in \PTC\
resemble LEGO blocks. They are self-contained units with three \emph{reference
frames}, one on the face where  particles enter the LEGO-block element, one
in the middle of the block, and one on the face where particles exit the block.

  \item[lens]
A synonym for magnet.

  \item[{\textls[50]{LHC}}]
See \emph{Large Hadron Collider.}

  \item[{\textls[50]{MAD}}]
Methodical Accelerator Design.

  \item[misalignment]
In \PTC, a mechanism for tracking from a beamline element positioned
where we want it to be in the lattice to the same beamline element
when it is out of alignment, that is, not positioned where we want
it to be.

  \item[node layout]
A data type containing a linked list of integration nodes, which
represents an expanded beam line.

  \item[normal form]
A formal statement concerning the stability properties of an accelerator ring.

  \item[one-turn map]
A map that tracks particles through one complete turn around a recirculating
accelerator. \PTC\ separates the creation of the one-turn map from the
analysis of the one-turn map.

  \item[patch]
A data type that connects the exit reference frame of one beamline
element to the entrance reference frame of another element when both
elements are positioned where we want them to be in the lattice.

  \item[phase-space variable]
``Within the context of a model system in classical mechanics,
the phase space coordinates of the system at any given time are composed
of all of the system's dynamical variables. Because of this, it is
possible to calculate the state of the system at any given time in
the future or the past, through integration of Hamilton's or Lagrange's
equations of motion.'' From Wikipedia.

   \item[polymorph]
A \Fninety\ type, \ptc{real\_8}, that can be a real number, a Taylor series,
or a special type of Taylor series called a \emph{knob}. \PTC's tracking routines track
polymorphs (thus the name Polymorphic Tracking Code).

  \item[polymorphic block]
To be provided.

  \item[probe]
A data type that holds information about the location of a particle.

  \item[\PTC]
See \emph{Polymorphic Tracking Code.}

  \item[Polymorphic Tracking Code]
A library of magnet routines programmed in \Fninety\ that makes use
of \FPP\ (Fully Polymorphic Package). \PTC\ implements symplectic
integration and provides structures for the description of a beam line.

  \item[reference frame]
A local coordinate system that \PTC\ uses to track the orientation and
phase space as particles enter and exit a beamline element. Each
element has three reference frames: one on the face where particles
enter the element, one in the middle of the element, and one on the face
where particles exit the element.

  \item[reference orbit]
See \emph{reference trajectory.}

  \item[reference trajectory]
A trajectory or orbit that describes the ideal path of a particle through
an ideal accelerator. Because ideal accelerators do not exist and a 
reference trajectory fails to account for misalignments and other errors
in an actual machine, \PTC\ does not use a reference trajectory.

  \item[Relativistic Heavy Ion Collider]
A heavy-ion collider operated by Brookhaven National Laboratory.

  \item[{\textls[50]{RHIC}}]
See \emph{Relativistic Heavy Ion Collider.}

  \item[resonance]
My memory jogger: ``repetitive 'kicking' caused by instability
in an accelerator''.

  \item[$s$-based tracking]
Tracking particles using the $s$-variable in a cylindrical coordinate system
(the $z$-variable in a Cartesian coordinate system) as the independent variable.
Also called \emph{magnet}-based tracking.

  \item[siamese]
Beamline elements that are joined and can be moved as a group.

  \item[source file]
The \PTC\ file in which a user defines an accelerator model, tracks
it to locate the beam line, and calculates global and local information.

  \item[state variable]
See \emph{phase-space variable.}

  \item[survey mode]
In \PTC, a command that locates the accelerator's complete beam line
by automatically tracking element by element through the beam line.

  \item[temporal beam]
To be provided.

  \item[temporal probe]
A data type that holds a probe and also information relevant to time-based tracking.

  \item[thin lens]
See \emph{integration node.}

  \item[thin layout]
See \emph{node layout.}

  \item[topology]
See \emph{accelerator topology.}

  \item[{\textls[50]{TPSA}}]
Truncated Power Series Algebra.

  \item[tune]
The number of times that a particle oscillates in one trip around
an accelerator. For example, a tune might be 7.14 oscillations. A tune
is associated with the x-axis (going sideways), and another tune is
associated with the y-axis (going up and down).

  \item[Twiss parameters]
To be provided. Local property.
\end{description}

\endinput
