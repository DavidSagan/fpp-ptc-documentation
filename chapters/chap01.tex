%!TEX root = ../PTC-LibUG.tex

\cleardoublepage
\chapter{Introduction}
\label{cha:intro}


\section{\PTC\ and \FPP}

\index{PTC!defined}
\index{Polymorphic Tracking Code|see{\PTC}}
%
The Polymorphic Tracking Code (\PTC) is a library of \Fninety\ data
structures and subroutines for integrating the equations of orbital
and spin motion for particles in modern accelerators and storage rings.
The data structures hold the material that we will mould into
physically correct and consistent three-dimensional computer models
of the complex topologies found in machines such as colliders and
recirculating linacs, including the effects of errors in both
location and strength of the various machine elements.
We use the subroutines to actually construct the computer model
of a machine and then integrate through it---including the
geometric transformations (translations and rotations) of machine
elements, connecting elements to form trackable beamlines, and
tracking the orbital and spin degrees of freedom.
In a word, \PTC\ is an \emph{integrator}.

\index{FPP!defined}
\index{Fully Polymorphic Package|see{\FPP}}
\index{normal form!FPP analysis tool}
\index{Taylor type!polymorphic}
\index{Poincar\'e map!FPP}
\index{map|see{one-turn map, Poincar\'e map, \emph{or} Taylor map}}
%
The Fully Polymorphic Package (\FPP)
%\cite{Forest:2006:FPPDoc}
is a package of polymorphic types and tools that provide \PTC\ with
facilities for analysis. In particular, \FPP\ implements a polymorphic
Taylor type (hence the \emph{P} in \PTC) that can change shape at
execution time. This Taylor type makes it possible for \FPP\ to extract
a Poincar\'e map from \PTC\ (or some other integrator). Moreover,
\FPP\ provides the tools to analyse the resulting map. The most
common---and most important---tool is the normal form: with
this at hand, one can compute tunes, lattice functions,
and nonlinear extensions of these and all other standard quantities of
accelerator theory. Indeed, the combination of \PTC\ and \FPP\ gives
access to all of standard perturbation theory on complicated
accelerator lattice designs.

\index{PTC!features}
\index{Bengtsson, Johan!pioneering work}
\index{Taylor map!derived from integrator}
\index{integrator!Taylor map}
%
In a nutshell, then, the three central features of \PTC%
\sidenote{When we say ``\PTC'', we shall (almost always) mean \PTC\ and
\FPP\ combined; but you keep in mind the distinction.}
 are
\begin{itemize}
  \item the \Fninety\ types and code that facilitate fully three-dimensional
        placement of beamline elements and construction of non-simple
        accelerator topologies: colliders, recirculators, dog-bones,
        and combinations of these topologies;
  \item the polymorphic integration routines for orbit and spin;
  \item the use of maps derived from the integrator---via operator
        overloading and polymorphism---for computing the full range
        of accelerator properties, including the parameter dependence
        of these properties.
\end{itemize}

\subsection{History of \PTC}

\fxnote{Check this, and fill in details.}
Latter half of 1980s, at SSC-CDG: Berz introduces automatic
differentiation and develops \DA\ package in \Fsevens.
Forest builds \LieLib\ on top of \DA, thus bringing the power
of map methods to the analysis of maps produced by integrators.
Early 1990s: J. Bengtsson pioneers the use of run-time polymorphism
to greatly simplify and improve the process of making run-time
changes to, for example, computing parameter dependence.
Forest and Bengtsson together develop the ideas from which derive
the fundamental building blocks of \PTC. Forest develops the
dynamical Euclidean group.
Middle to late 1990s: Forest develops \LieLib\ into \FPP\
(using \LBNL\ versions of \DA\ and \LieLib).
\FPP\ replaces the real variable type \ptc{real(8)} with a new type
called \ptc{real\_8} to produce Taylor series for analysis.
Forest also develops \PTC-proper (the integrator).
Together these tools compose \PTC.
Middle to late 2000s: Forest adds spin dynamics to \PTC.
L. Yang develops a \Cpp\ replacement for Berz's \DA\ package.
\et{ Recently Forest has added, to \FPP\,   a complex analysis packaged based on a ``complexification'' of Berz's \DA\ package. 
This tool is compatible with the existing \PTC\ and it is certainly easier to use because its  programming style looks very much like the theory. }

\subsection{Where to Obtain \PTC}

\et{ The code \PTC\ and the library \FPP\ can be downloaded by following the instructions at the following URL. They can also be downloaded at the BMAD website. }

%\fxnote{\'Etienne: Is this the right place for people to go?}
\url{http://www.takafumi.org/From_Tracking_Code_to_Analysis}

\et{ \noindent If you download \PTC\ from the instruction at that URL, you will have access to examples which can help you with the  new complexified \FPP. Forest is actually writing a book which uses these examples. The example are a free download.  }

%\subsection{How to Install \PTC}
%
%To be provided.


\section{\PTC\ Library User Guide}

%This section describes the goals of the \emph{\PTC\ Library User Guide}
%and offers suggestions on where to start reading.
%
%
%\subsection{Goals}
%
%This \emph{\PTC\ Library User Guide}
%\begin{itemize}
%  \item explains the advantages of using \PTC\ to model an accelerator;
%  \item explains \PTC\ accelerator modeling concepts and techniques;
%  \item provides tutorial examples of \PTC\ code;
%  \item documents the Tech-X space-charge library.
%\end{itemize}
%

\subsection{Where to Start Reading}

The \TCref{overview} presents concepts important for understanding \PTC.
In-depth discussions of all these concepts can be found in several
publications by \'Etienne Forest:
\begin{itemize}
  \item \'Etienne Forest,
``A Hamiltonian-free description of single particle dynamics for
hopelessly complex periodic systems'',
\emph{J. Math.\ Phys.}, 31(5):1133--1144, May 1990,
\doi{10.1063/1.528795};

  \item \'Etienne Forest and Kohji Hirata,
``A contemporary guide to beam dynamics'',
Technical Report KEK-92-12, KEK, Tsukuba, Japan, August 1992;

  \item \'Etienne Forest,
``Locally accurate dynamical Euclidean group'',
\emph{Phys.\ Rev.\ E}, 55(4):4665--4674, April 1997,
\doi{10.1103/PhysRevE.55.4665};

  \item \'Etienne Forest,
\emph{Beam Dynamics: A New Attitude and Framework},
volume 8 of
\emph{The Physics and Technology of Particle and Photon Beams},
Harwood Academic Publishers, Amsterdam, 1998;

  \item \'Etienne Forest,
``Geometric integration for particle accelerators'',
\emph{J. Phys.\ A: Math.\ Gen.}, 39(19):5321--5377, May 2006,
\doi{10.1088/0305-4470/39/19/S03}.
\end{itemize}

%\cite{Forest:1990:HamFree,%
%      Forest:1992:ContGuide,%
%      Forest:1997:DynEuclid,%
%      Forest:1998:BeamDyn,%
%      Forest:2006:GeomIntegPA}

If you want to get started right away, begin with \CTref{model.accel}
and refer back to the overview as needed while working with the tutorial
examples.

\endinput
