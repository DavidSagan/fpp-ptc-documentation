%!TEX root = ../PTC-LibUG.tex

\chapter{Tracking Routines}
\label{cha:tracking}

\index{routines!tracking}
\index{tracking routines!list of}
%
\PTC's tracking routines are divided into four categories:

\begin{itemize}
  \item standard tracking routines on fibres,
  \item tracking routines on integration nodes,
  \item tracking routines on 3-D information through an integration node,
  \item time-based tracking routines.
\end{itemize}

A fifth section documents the closed-orbit routine.

Mandatory arguments for the tracking routines are in
\ptc{regular black} type. Optional arguments are in \ptc{\textit{\textcolor{red}{red italic}}} type.

\index{one-turn map!creating}
%
Positions are normally specified by \ptc{I1}, \ptc{I2}, \ptc{fibre1},
\ptc{fibre2}, \ptc{node1}, or \ptc{node2}. Generally, if only
\ptc{x1} is present (\ptc{x=I}, \ptc{fibre}, or \ptc{node}), this produces a
one-turn map from position \ptc{x1} back to position \ptc{x1} if the layout
is closed. If the layout is opened, then it goes to the end of the line.

If \ptc{x1} and \ptc{x2} are present, the routine tracks through \ptc{x1} all
the way to the front of \ptc{x2} (\ptc{x2} not included).

The only exception to all of this is the time-based tracking routine.

\fxnote{Al: \'Etienne mentioned that we should discuss spinors in relation to tracking routines.}


\section{Standard Tracking Routines on Fibres}

\index{routines!standard tracking}
\index{tracking routines!standard}
\index{routines!tracking fibres}
\index{tracking routines!fibres}
\index{spin!tracking routines on fibres}
\index{radiation!tracking routines on fibres}
\index{tracking routines!spin}
\index{tracking routines!radiation}
\index{fibre!tracking routines}
%
These routines do not support spin and radiation.


\subsection{Track}

\index{track@\ptc{track}!routine}
\index{routine!\ptc{track}}
%
\begin{ptccode}
track (R, X\textit{\textcolor{red}{, I1, I2}}, K)
\end{ptccode}

\ptc{X} is an array of six \ptc{real(dp)} or \ptc{REAL_8}.

\ptc{I1} and \ptc{I2} are the position of the fibres.

\ptc{Result=TRACK_FLAG (R,X,I1,I2,k)}

\ptc{Result=logical}; true indicates stable; false indicates unstable.

\begin{ptccode}
track (C, X, K\textit{\textcolor{red}{, CHARGE}})
\end{ptccode}

This routine tracks through the fibre \ptc{C}.


\subsection{Find_orbit}

\index{find_orbit@\ptc{find_orbit}!routine}
\index{routine!\ptc{find_orbit}}
%
\begin{ptccode}
find_orbit(R, FIX, LOC, STATE, eps\textit{\textcolor{red}{, TURNS}})
\end{ptccode}

The \ptc{find_orbit} routine works in the same way as the
subroutine \ptc{find_orbit_x} but on the fibre structure
without radiation. For more information about the subroutine
\ptc{find_orbit_x}, see \Tref{sub:Find-Orbit-X}.

\ptc{LOC} is the integer location in the layout \ptc{R}.


\section{Tracking Routines on Integration Nodes}

\index{routines!tracking integration nodes}
\index{tracking routines!integration nodes}
\index{spin!tracking routines on integration nodes}
\index{radiation!tracking routines on integration nodes}
\index{tracking routines!spin}
\index{tracking routines!radiation}
\index{integration node!tracking routines}
%
These routines support spin and radiation.


\subsection{Routines for Tracking either Probe or Probe_8}

\index{probe!tracking routines}
%
For the type definitions of \ptc{probe} and \ptc{probe_8}, see
\Tref{sub:Probe-B}.


\subsubsection{TRACK_PROBE2}

\index{TRACK_PROBE2@\ptc{TRACK_PROBE2}!routine}
\index{routine!\ptc{TRACK_PROBE2}}
%
\begin{ptccode}
TRACK_PROBE2(R,XS,K\textit{\textcolor{red}{,I1,I2}})
\end{ptccode}

\ptc{I1,I2 = NODE POSITION}

\ptc{I1} only implies a one-turn map.

\ptc{K=INTERNAL STATE}


\subsubsection{TRACK_PROBE}

\index{TRACK_PROBE@\ptc{TRACK_PROBE}!routine}
\index{routine!\ptc{TRACK_PROBE}}
%
\begin{ptccode}
TRACK_PROBE (R,XS,K\textit{\textcolor{red}{,FIBRE1,FIBRE2,NODE1,NODE2}})
\end{ptccode}

\ptc{FIBRE1,FIBRE2,NODE1,NODE2} are all integer positions of either the
fibre or the integration node.


\subsubsection{TRACK_NODE_PROBE}

\index{TRACK_NODE_PROBE@\ptc{TRACK_NODE_PROBE}!routine}
\index{routine!\ptc{TRACK_NODE_PROBE}}
%
\begin{ptccode}
TRACK_NODE_PROBE(T,XS,K)
\end{ptccode}

\ptc{T} is an integration node.


\subsubsection{Object-Oriented Routines}

\index{routines!object-oriented}
\index{tracking routines!object-oriented}
\index{TRACK_PROBE2@\ptc{TRACK_PROBE2}!routine}
\index{routine!\ptc{TRACK_PROBE2}}
\index{TRACK_PROBE@\ptc{TRACK_PROBE}!routine}
\index{routine!\ptc{TRACK_PROBE}}
\index{TRACK_NODE_PROBE@\ptc{TRACK_NODE_PROBE}!routine}
\index{routine!\ptc{TRACK_NODE_PROBE}}
%
\begin{ptccode}
TRACK_PROBE2 (XS,K\textit{\textcolor{red}{,FIBRE1,FIBRE2,NODE1,NODE2}})
TRACK_PROBE (XS,K\textit{\textcolor{red}{,FIBRE1,FIBRE2,NODE1,NODE2}})
TRACK_NODE_PROBE (XS,K\textit{\textcolor{red}{,FIBRE1,FIBRE2,NODE1,NODE2}})
\end{ptccode}

These are all calls to the same routine. The fibres and the nodes
are actual pointers to the objects.

\index{one-turn map!tracking}
%
One turn can be done as follows:

\begin{ptccode}
TRACK_PROBE (XS,K\textit{\textcolor{red}{,NODE1=T,NODE2=T%PREVIOUS}})
\end{ptccode}

\index{routine!\ptc{TRACK_PROBE_X}}
\index{TRACK_PROBE_X@\ptc{TRACK_PROBE_X}!routine}
%
In the \ptc{TRACK_PROBE_X} routine, the fibres and the nodes are pointers to the objects.
The routine wraps the \ptc{TRACK_PROBE} routine shown above.

\begin{ptccode}
TRACK_PROBE_X (R,X,K,U,T\textit{\textcolor{red}{,FIBRE1,FIBRE2,NODE1,NODE2}})
\end{ptccode}

\fxnote{Al: Is my rewording above correct? What does ``wrap'' mean? Will the meaning be clear to our audience?}


\subsection{Routines for Tracking either Real or Real_8}

All these routines wrap the above routines and therefore support radiation,
beam-beam and $s$-dependent apertures.


\subsubsection{TRACK_NODE_X}

\index{routine!\ptc{TRACK_NODE_X}}
\index{TRACK_NODE_X@\ptc{TRACK_NODE_X}!routine}
%
\begin{ptccode}
TRACK_NODE_X(T,X,K)
\end{ptccode}

\subsubsection{TRACK_PROBE_X}

\index{routine!\ptc{TRACK_PROBE_X}}
\index{TRACK_PROBE_X@\ptc{TRACK_PROBE_X}!routine}
%
\begin{ptccode}
TRACK_PROBE_X(R,X,K\textit{\textcolor{red}{,U,T,FIBRE1,FIBRE2,NODE1,NODE2}})
\end{ptccode}

\ptc{U=LOGICAL} where \ptc{TRUE} indicates \ptc{UNSTABLE} (optional).

\ptc{T} is a pointer to the fibre where the particle is lost (optional).


\subsubsection{TRACK_BEAM}

\index{routine!\ptc{TRACK_BEAM}}
\index{TRACK_BEAM@\ptc{TRACK_BEAM}!routine}
\index{tracking routines!beam of particles}
%
This routine tracks a beam of particles.

\begin{ptccode}
TRACK_BEAM(R,B,K\textit{\textcolor{red}{,T,FIBRE1,FIBRE2,NODE1,NODE2}})
\end{ptccode}

For the type definition of \ptc{BEAM}, see Section~\ref{sub:Beam-B}.


\section{Tracking Routines on 3-D Information through an Integration Node}

\index{routines!tracking integration nodes}
\index{tracking routines!integration nodes}
\index{tracking routines!3-D information through integration node}
\index{integration node!tracking routines}
%
These routines track three-dimensional information through an integration node.


\subsection{Track_node_v}

\index{routine!\ptc{TRACK_NODE_V}}
\index{TRACK_NODE_V@\ptc{TRACK_NODE_V}!routine}
%
The \ptc{TRACK_NODE_V} routine tracks a trajectory and records its
three-dimensional position for plotting at the beginning and the end of a node.

\begin{ptccode}
TRACK_NODE_V (T,V,K,REF)
\end{ptccode}

\ptc{T} is an integration node.

\ptc{V} is of type \ptc{THREE_D_INFO}. The type definition is given below.

\index{routine!\ptc{TRACK_FILL_REF}}
\index{TRACK_FILL_REF@\ptc{TRACK_FILL_REF}!routine}
%
\ptc{REF=TRUE} or \ptc{FALSE}. If \ptc{REF=TRUE}, then the results of the
\ptc{TRACK_FILL_REF} routine are used. The ray is magnified by 
\ptc{V\%SCALE} (see type \ptc{THREE_D_INFO} below) with respect to a
trajectory computed and stored by the \ptc{TRACK_FILL_REF} routine,
which tracks the ray \ptc{FIX} from fibre \ptc{I1} back to fibre \ptc{I1}.

\begin{ptccode}
TRACK_FILL_REF(R,FIX,I1,K))
\end{ptccode}

\index{data type!\ptc{THREE_D_INFO}}
\index{THREE_D_INFO@\ptc{THREE_D_INFO}!data type}
\index{three-dimensional information!data type}
\index{3-D information|see{three-dimensional information}}
%
Here is the data type definition for three-dimensional information.

\begin{ptccode}
TYPE THREE_D_INFO
  REAL(DP) A(3),B(3)         ! CENTRE OF ENTRANCE AND EXIT FACES
  REAL(DP) ENT(3,3),EXI(3,3) ! ENTRANCE AND EXIT FRAMES FOR DRAWING MAGNET FACES
  REAL(DP) WX,WY             ! WIDTH OF BOX FOR PLOTTING PURPOSES
  REAL(DP) O(3),MID(3,3)     ! FRAMES AT THE POINT OF TRACKING
  REAL(DP) REFERENCE_RAY(6)  !
  REAL(DP) X(6)              ! RAY TRACKED WITH REFERENCE_RAY USING A TYPE(BEAM)
  REAL(DP) R0(3),R(3)        ! RAY POSITION GLOBAL RETURNED
  REAL(DP) SCALE             ! MAGNIFICATION USING REFERENCE_RAY
  LOGICAL(LP) U(2)           ! UNSTABLE FLAG FOR BOTH RAY AND REFERENCE_RAY
END TYPE THREE_D_INFO
\end{ptccode}


\section{Time-based Tracking Routines}

\index{routines!time-based tracking}
\index{tracking routines!time-based}
%
These routines provide time-based tracking of temporal probes and temporal beams.


\subsection{Track_time}

\index{routine!\ptc{TRACK_TIME}}
\index{TRACK_TIME@\ptc{TRACK_TIME}!routine}
%
\begin{ptccode}
TRACK_TIME(XT,DT,K)
\end{ptccode}

\ptc{XT} is a temporal probe. For the type definition of \ptc{TEMPORAL_PROBE},
see Section~\ref{sub:Temporal-Probe-B}.


\subsection{Track_temporal_beam}

\index{routine!\ptc{TRACK_TEMPORAL_BEAM}}
\index{TRACK_TEMPORAL_BEAM@\ptc{TRACK_TEMPORAL_BEAM}!routine}
%
\begin{ptccode}
TRACK_TEMPORAL_BEAM(B,DT,STATE)
\end{ptccode}

\ptc{B} is a temporal beam. For the type definition of \ptc{TEMPORAL_BEAM},
see Section~\ref{sub:Temporal-Beam-B}.


\section{Closed-Orbit Routine}

\index{closed orbit!finding}
%
This routine finds the closed orbit.


\subsection{Find_orbit_x}
\label{sub:Find-Orbit-X}

\index{routine!\ptc{FIND_ORBIT_X}}
\index{FIND_ORBIT_X@\ptc{FIND_ORBIT_X}!routine}
%
\begin{ptccode}
FIND_ORBIT_X(R,FIX,STATE,eps\textcolor{red}{,TURNS,fibre1,node1})
\end{ptccode}

Here \ptc{fibre1} and \ptc{node1} are integer positions. The routine finds the
fixed point for \ptc{TURNS} turns; one turn if not specified.

The argument \ptc{eps= real(dp)} number is used to do numerical
differentiation---typically \ptc{1.d-6} works.

For a no-cavity fixed point, \ptc{fix(5)} must contain the energy variable.

\endinput
