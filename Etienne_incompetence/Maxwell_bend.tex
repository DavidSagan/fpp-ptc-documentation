% !TEX TS-program = pdflatex
% !TEX encoding = UTF-8 Unicode

% This is a simple template for a LaTeX document using the "article" class.
% See "book", "report", "letter" for other types of document.

\documentclass[11pt]{article} % use larger type; default would be 10pt

\usepackage[utf8]{inputenc} % set input encoding (not needed with XeLaTeX)

%%% Examples of Article customizations
% These packages are optional, depending whether you want the features they provide.
% See the LaTeX Companion or other references for full information.

%%% PAGE DIMENSIONS
\usepackage{geometry} % to change the page dimensions
\geometry{a4paper} % or letterpaper (US) or a5paper or....
% \geometry{margin=2in} % for example, change the margins to 2 inches all round
% \geometry{landscape} % set up the page for landscape
%   read geometry.pdf for detailed page layout information

\usepackage{graphicx} % support the \includegraphics command and options

% \usepackage[parfill]{parskip} % Activate to begin paragraphs with an empty line rather than an indent

%%% PACKAGES
\usepackage{booktabs} % for much better looking tables
\usepackage{array} % for better arrays (eg matrices) in maths
\usepackage{paralist} % very flexible & customisable lists (eg. enumerate/itemize, etc.)
\usepackage{verbatim} % adds environment for commenting out blocks of text & for better verbatim
\usepackage{subfig} % make it possible to include more than one captioned figure/table in a single float
% These packages are all incorporated in the memoir class to one degree or another...

%%% HEADERS & FOOTERS
\usepackage{fancyhdr} % This should be set AFTER setting up the page geometry
\pagestyle{fancy} % options: empty , plain , fancy
\renewcommand{\headrulewidth}{0pt} % customise the layout...
\lhead{}\chead{}\rhead{}
\lfoot{}\cfoot{\thepage}\rfoot{}

%%% SECTION TITLE APPEARANCE
\usepackage{sectsty}
\allsectionsfont{\sffamily\mdseries\upshape} % (See the fntguide.pdf for font help)
% (This matches ConTeXt defaults)

%%% ToC (table of contents) APPEARANCE
\usepackage[nottoc,notlof,notlot]{tocbibind} % Put the bibliography in the ToC
\usepackage[titles,subfigure]{tocloft} % Alter the style of the Table of Contents
\renewcommand{\cftsecfont}{\rmfamily\mdseries\upshape}
\renewcommand{\cftsecpagefont}{\rmfamily\mdseries\upshape} % No bold!


\def\th#1{theorem~\ref{#1}}
\def\Th#1{Theorem~\ref{#1}}
\def\eq#1{Eq.~(\ref{#1})}
\def\Eq#1{Eq.~(\ref{#1})}
\def\eqs#1{Eqs.~(\ref{#1})}
\def\eqe#1{(\ref{#1})}
\def\sec#1{Sec.~(\ref{#1})}
\def\secs#1{Secs.~(\ref{#1})}
\def\sece#1{(\ref{#1})}
\def\Fig#1{Figure {\rm \ref{#1}}}
\def\fig#1{Figure {\rm \ref{#1}}}
\def\fige#1{ {\rm \ref{#1}}}
\def\figs#1{Figures {\rm \ref{#1}}}
\def\chas#1{Chapters {\rm \ref{#1}}}
\def\Chas#1{Chapters {\rm \ref{#1}}}
\def\app#1{Appendix {\rm \ref{#1}}}
\def\cha#1{Chapter {\rm \ref{#1}}}
\def\chasup#1{supplemental chapter {\rm \ref{#1}}}
\def\Cha#1{Chapter {\rm \ef{#1}}}
\def\p#1{\pageref{#1}}
%%% The "real" document content comes below...
%\font\tencafm=Arialr at 9.0pt                      
\def\hel{\rm }
\def\appfont{\scriptsize }
%1.\tiny 
%2.\scriptsize 
%3.\footnotesize 
%4.\small 
%5.\normalsize 
%6.\large 
%7.\Large 
%8.\LARGE 
%9.\huge 
%10.\Huge 
\newcommand{\dragt}{\mbox{\scriptsize $\begin{matrix}\# \\
\# \end{matrix}$}}
 
%%% END Article customizations

%%% The "real" document content comes below...

\title{Maxwell Equations  in PTC's Sector Bend}
\author{E. Forest}
%\date{} % Activate to display a given date or no date (if empty),
         % otherwise the current date is printed 

\begin{document}
\maketitle

\section{Magnetic field}


We will say that a bend has sector geometry if magnetic field and/or the electric field is invariant as we travel on a circle of radius $r$ down the center of the
element. The symmetry would be perfect if the entire element was a cyclotron; so here we neglect fringe fields at both ends.

In the case of a magnetic field, the vector potential $a_s$ must obey the equation:
%
%
%]|Expr|[#b @`b___}):# b'4" Chicago^: ;bP8&c0!*,^<c!(A($$^"!Symbol^:!&c0  /1|
%|^->^""*|:"&c0!*2,M<2^h(#1,Khx}}<2(&:!&c0  .V"#Helvetica|:#&c0!*-<|
%|hphantom-[:"x:#-]}(":!&c0  .V:"&c0!*x}}}}F:#,F,]0"$Monaco^:$ |
%| where  :#F,]<c!$1(#:"r,Kx}}$^:#a^s_:$ and "%Times|:%h,]<2^1^|
%|r}: ;8/<:";bP8eq,Zmaxb: ;8/=;bP8,\}& b!( b"0 b#8 b$@ b%H b&P!WW}]|[
\begin{eqnarray}\left\{{{\nabla }_{\perp }^{2}-{h \over 1+hx}{\partial \hphantom{x} \over \partial x}}\right\}F&=0~~{\rm w}{\rm h}{\rm e}{\rm r}{\rm e}~~F=\left({r+x}\right){a}_{s}~{\rm a}{\rm n}{\rm d} ~h=
{1 \over r}\label{eq:maxb}\end{eqnarray}
%
%
%
%]|Expr|[#b @`b___})## b'4" Helvetica|: ;bP8&c0!*,D$^"!Symbol^:!&c0  /1|
%|^->^""*|:"&c0!*2: ,D}& b!( b"0 b#8 b$@ b%H b&P!WW}]|[
${\nabla }_{\perp }^{2}$
is the transverse Laplacian %
%]|Expr|[#b @`b___})%# b'4" Helvetica|: ;bP8&c0!*,D<2(&$^"!Symbol^:!&c0  .V|
%|_^""*|:"&c0!*2: -<hphantom-[:"x: -]}^$(":!&c0  .V:"&c0!*x}_^2}|
%|,K<2(&$^:!&c0  .V_^:"&c0!*2: -<hphantom-[:"x: -]}^$(":!&c0  .V|
%|:"&c0!*y}_^2}: ,D}& b!( b"0 b#8 b$@ b%H b&P!WW}]|[
${{\partial }^{2}\hphantom{x} \over {\partial x}^{2}}+{{\partial }^{2}\hphantom{x} \over {\partial y}^{2}}$.

In a straight element, where %
%]|Expr|[#b @`b___})%# b'4" Helvetica|: ;bP8&c0!*,D"!*|:!r,]""Symbol^:"&c0  .E|
%|: &c0!*,D}& b!( b"0 b#8 b$@ b%H b&P!WW}]|[
$r=
\infty $,
the solutions if \eq{eq:maxb} used in PTC, are just:
%
%]|Expr|[#b @`b___})3# b'4" Chicago^: ;bP8&c0!*,^$^"!*|:!F^n(#|
%|h,]0}""Helvetica|:",F,],M<2^$^b^n_^n}Re<c!(A^$^<c!$1(#x,Kiy}}|
%|_^n},K<2^$^a^n_^n}Im<c!(A^$^<c!$1(#x,Kiy}}_^n}  : ;8/<:!;bP8eq|
%|,Zhsol: ;8/=:";bP8 : ,\}& b!( b"0 b#8 b$@ b%H b&P!WW}]|[
\begin{eqnarray}{F}_{n}^{h=
0}&=-{{b}_{n} \over n}Re\left\{{{\left({x+iy}\right)}^{n}}\right\}+{{a}_{n} \over n}Im\left\{{{\left({x+iy}\right)}^{n}}\right\}  \label{eq:hsol} \end{eqnarray}
%
%
These solutions have the property that, in the midplane $y=0$,  the magnetic field is given by:
%
%]|Expr|[#b @`b___})7# b'4" Chicago^: ;bP8&c0!*,^$^"!*|:!b^y_,]|
%|<c%#D("$^b^n_$^x_(#n,M1}}("""Helvetica|:"-<sum}^:!n_}"#Monaco^:# |
%|  and   $^:!b^x_,]<c%#D("$^a^n_$^x_(#n,M1}}(":"-<sum}^:!n_}:" |
%|: ;8/<:!;bP8eq,Zst,Zanbn: ;8/=;bP8,\}& b!( b"0 b#8 b$@ b%H b&P!WW}]|[
\begin{eqnarray}{b}_{y}=
\sum\limits_{n}^{} {b}_{n}{x}^{n-1}~~~{\rm a}{\rm n}{\rm d} ~~~{b}_{x}=
\sum\limits_{n}^{} {a}_{n}{x}^{n-1} \label{eq:st:anbn}\end{eqnarray}
%
%
Of course the full solution is given by an infinite sum:
%
%]|Expr|[#b @`b___}),# b'4" Chicago^: ;bP8&c0!*,^$^"!*|:!F^n^h|
%|""Symbol^:" "#Helvetica|:#,F:!,]:#,F:" <c%#D)!# b'4$^&c0  D^:!&c0!*n|
%|^k}& b!( b"0 b#8 b$@ b%H b&P!WW}(":#-<sum}(%:!k,]0,L:"&c0  .E}|
%|_}: ;8&c0!*/<:!;bP8fnk: ;8/=;bP8,\}& b!( b"0 b#8 b$@ b%H b&P!WW}]|[
\begin{eqnarray}{F}_{n}^{h} &=
& \sum\limits_{k=
0,\infty }^{} {\Delta }_{n}^{k}\label{fnk}\end{eqnarray}
%
%
The idea is to set up a recursive relation to solve for the $F_n^k$ while continuously respecting \eqs{eq:maxb} and \eqe{eq:st:anbn}. Let us see how that works at order $k$. 
%
%]|Expr|[#b @`b___})5# b'4" Chicago^: ;bP8&c0!*,^$^"!Symbol^:!&c0  D|
%|^""*|:"&c0!*n(#k,K1}:! "#Helvetica|:#,F:",]:#,F$^:!&c0  /1^->|
%|(":"&c0!*,M2}<c%#D)## b'4<2^h(#1,Khx}}<2(&:!&c0  .V:#&c0!*-<hphantom|
%|-[:"x:#-]}(":!&c0  .V:"&c0!*x}}$^:!&c0  D^:"&c0!*n^:!&c0  a|
%|}& b!( b"0 b#8 b$@ b%H b&P!WW}(":#&c0!*-<sum}(%:!&c0  a:"&c0!*,]|
%|0,Lk}_},K$^C(#k,K1}_$^F(#n,Kk}^0  : ;8/<:";bP8eq,Zsol,Zanbn: ;8/=|
%|;bP8,\}& b!( b"0 b#8 b$@ b%H b&P!WW}]|[
\begin{eqnarray}{\Delta }_{n}^{k+1} &=
&{\nabla }_{\perp }^{-2}\sum\limits_{\alpha =
0,k}^{} {h \over 1+hx}{\partial \hphantom{x} \over \partial x}{\Delta }_{n}^{\alpha }+{C}_{k+1}{F}_{n+k}^{0}\ \ \label{eq:sol:anbn}\end{eqnarray}
%
%
To solve this equation, one first initializes it at $k=0$ with 
%
%]|Expr|[#b @`b___})## b'4" Helvetica|: ;bP8&c0!*,D$^"!*|:!F^n|
%|(#h,]0}: ,D}& b!( b"0 b#8 b$@ b%H b&P!WW}]|[
${F}_{n}^{h=
0}$
of \eq{eq:hsol}. The operator %
${\nabla }_{\perp }^{-2}$ of \eq{eq:sol:anbn} find a solution for the inverse Laplacian. In phasors variable 
%
%]|Expr|[#b @`b___})%# b'4" Helvetica|: ;bP8&c0!*,D<c$%!^"!*|:!u}|
%|,]<c!$1('x,Kiy,Lx,Miy}}: ,D}& b!( b"0 b#8 b$@ b%H b&P!WW}]|[
$\vec{u}=
\left({x+iy,x-iy}\right)$
the inverse Laplacian is   given by 
%
%]|Expr|[#b @`b___})=# b'4" Chicago^: ;bP8&c0!*,^$^"!Symbol^:!&c0  /1|
%|^->("""*|:"&c0!*,M2}f,H<c$%!^u},I"#Helvetica|:#,F,]<c" #^<c" #|
%|(&f du d<c$%A^u}}_^<c$%A^u}}_^u} ,K C $^f_^0"$Monaco^:$  where|
%|  $^:!&c0  /1^->^:"&c0!*2$^:#f_^0,]0: ;8/<:#eq,Z;bP8inv: ;8/=|
%|;bP8,\}& b!( b"0 b#8 b$@ b%H b&P!WW}]|[
\begin{eqnarray}{\nabla }_{\perp }^{-2}f(\vec{u})&=\int_{}^{u}\int_{}^{\overline{u}}f du d\overline{u} + C {f}^{0}~~{\rm w}{\rm h}{\rm e}{\rm r}{\rm e}~~{\nabla }_{\perp }^{2}{f}^{0}=0\label{eq:inv}\end{eqnarray}
%

in \eq{eq:inv} the first term is an antiderivative with with respect to $u$ and $ \overline{u}$. The additive term is an harmonic solution of the type of \eq{eq:hsol}. 

To obtain the solution to the order of truncation on iterates \eq{eq:sol:anbn} starting with the harmonic solution %
%]|Expr|[#b @`b___})%# b'4" Helvetica|: ;bP8&c0!*,D$^"!Symbol^:!&c0  D|
%|^""*|:"&c0!*n(#k,]0},]$^F^n^0: ,D}& b!( b"0 b#8 b$@ b%H b&P!WW}]|[
${\Delta }_{n}^{k=
0}=
{F}_{n}^{0}$.

The only issue is how does one set the constant ${C}_{k+1}$? One notices that the harmonic solution  ${F}_{n}^{0}$ already obeys \eq{eq:st:anbn}. Therefore it suffices to  choose ${C}_{k+1}$ so as to remove leading order terms of the form $C x^{n+k+1}$  in ${\Delta }_{n}^{k+1}$ by adding a term proportional to ${F}_{n}^{k+1}$. Higher order terms which violate \eq{eq:st:anbn} while be handled at the next iterations.


\section{Electric Field}

The electric field should also obey Maxwell's equation. If we represent it by a potential $F$, its Laplacian must vanish. In cylindrical coordinates, it is given by the equation:
%
%]|Expr|[#b @`b___}):# b'4" Chicago^: ;bP8&c0!*,^$^"!Symbol^:!&c0  /1|
%|^->^""*|:"&c0!*2F:! "#Helvetica|:#,F:",]:#,F:! <c!(A)&# b'4<2|
%|^:"1(#1,Khx}}<2(&:!&c0  .V:#&c0!*-<hphantom-[:"x:#-]}(":!&c0  .V|
%|:"&c0!*x}}<c!$1(#1,Khx}}<2(&:!&c0  .V:#&c0!*-<hphantom-[:"x:#-]}|
%|(":!&c0  .V:"&c0!*x}},K<2(&$^:!&c0  .V_^:"&c0!*2:#-<hphantom-[|
%|:"x:#-]}^$(":!&c0  .V:"&c0!*y}_^2}}& b!( b"0 b#8 b$@ b%H b&P!WW}}|
%|F: ;bP;/":#;bP8 :! :#,F:",]:#,F:! <c!(A($$^&c0  /1^->^:"&c0!*2|
%|,K<2^h(#1,Khx}}<2(&:!&c0  .V:#&c0!*-<hphantom-[:"x:#-]}(":!&c0  .V|
%|:"&c0!*x}}}}F : ;8/<:#;bP8eq,Zelec: ;8/=;bP8,\|
%|}& b!( b"0 b#8 b$@ b%H b&P!WW}]|[
\begin{eqnarray}{\nabla }_{\perp }^{2}F &=
& \left\{{{1 \over 1+hx}{\partial \hphantom{x} \over \partial x}\left({1+hx}\right){\partial \hphantom{x} \over \partial x}+{{\partial }^{2}\hphantom{x} \over {\partial y}^{2}}}\right\}F\nonumber \\
  &=
& \left\{{{\nabla }_{\perp }^{2}+{h \over 1+hx}{\partial \hphantom{x} \over \partial x}}\right\}F\ \label{eq:elec}\end{eqnarray}
%



\end{document}
