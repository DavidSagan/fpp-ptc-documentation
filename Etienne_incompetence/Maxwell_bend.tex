% !TEX TS-program = pdflatex
% !TEX encoding = UTF-8 Unicode

% This is a simple template for a LaTeX document using the "article" class.
% See "book", "report", "letter" for other types of document.

\documentclass[11pt]{article} % use larger type; default would be 10pt

\usepackage[utf8]{inputenc} % set input encoding (not needed with XeLaTeX)
\usepackage[strings]{underscore}    % to use "_" in text
%%% Examples of Article customizations
% These packages are optional, depending whether you want the features they provide.
% See the LaTeX Companion or other references for full information.

%%% PAGE DIMENSIONS
\usepackage{geometry} % to change the page dimensions
\geometry{a4paper} % or letterpaper (US) or a5paper or....
% \geometry{margin=2in} % for example, change the margins to 2 inches all round
% \geometry{landscape} % set up the page for landscape
%   read geometry.pdf for detailed page layout information

\usepackage{graphicx} % support the \includegraphics command and options

% \usepackage[parfill]{parskip} % Activate to begin paragraphs with an empty line rather than an indent

%%% PACKAGES
\usepackage{booktabs} % for much better looking tables
\usepackage{array} % for better arrays (eg matrices) in maths
\usepackage{paralist} % very flexible & customisable lists (eg. enumerate/itemize, etc.)
\usepackage{verbatim} % adds environment for commenting out blocks of text & for better verbatim
\usepackage{subfig} % make it possible to include more than one captioned figure/table in a single float
% These packages are all incorporated in the memoir class to one degree or another...

%%% HEADERS & FOOTERS
\usepackage{fancyhdr} % This should be set AFTER setting up the page geometry
\pagestyle{fancy} % options: empty , plain , fancy
\renewcommand{\headrulewidth}{0pt} % customise the layout...
\lhead{}\chead{}\rhead{}
\lfoot{}\cfoot{\thepage}\rfoot{}

%%% SECTION TITLE APPEARANCE
\usepackage{sectsty}
\allsectionsfont{\sffamily\mdseries\upshape} % (See the fntguide.pdf for font help)
% (This matches ConTeXt defaults)

%%% ToC (table of contents) APPEARANCE
\usepackage[nottoc,notlof,notlot]{tocbibind} % Put the bibliography in the ToC
\usepackage[titles,subfigure]{tocloft} % Alter the style of the Table of Contents
\renewcommand{\cftsecfont}{\rmfamily\mdseries\upshape}
\renewcommand{\cftsecpagefont}{\rmfamily\mdseries\upshape} % No bold!


\def\th#1{theorem~\ref{#1}}
\def\Th#1{Theorem~\ref{#1}}
\def\eq#1{Eq.~(\ref{#1})}
\def\Eq#1{Eq.~(\ref{#1})}
\def\eqs#1{Eqs.~(\ref{#1})}
\def\eqe#1{(\ref{#1})}
\def\sec#1{Sec.~(\ref{#1})}
\def\secs#1{Secs.~(\ref{#1})}
\def\sece#1{(\ref{#1})}
\def\Fig#1{Figure {\rm \ref{#1}}}
\def\fig#1{Figure {\rm \ref{#1}}}
\def\fige#1{ {\rm \ref{#1}}}
\def\figs#1{Figures {\rm \ref{#1}}}
\def\chas#1{Chapters {\rm \ref{#1}}}
\def\Chas#1{Chapters {\rm \ref{#1}}}
\def\app#1{Appendix {\rm \ref{#1}}}
\def\cha#1{Chapter {\rm \ref{#1}}}
\def\chasup#1{supplemental chapter {\rm \ref{#1}}}
\def\Cha#1{Chapter {\rm \ef{#1}}}
\def\p#1{\pageref{#1}}
%%% The "real" document content comes below...
%\font\tencafm=Arialr at 9.0pt                      
\def\hel{\rm }
\def\appfont{\scriptsize }
%1.\tiny 
%2.\scriptsize 
%3.\footnotesize 
%4.\small 
%5.\normalsize 
%6.\large 
%7.\Large 
%8.\LARGE 
%9.\huge 
%10.\Huge 
\newcommand{\dragt}{\mbox{\scriptsize $\begin{matrix}\# \\
\# \end{matrix}$}}
 
%%% END Article customizations

%%% The "real" document content comes below...

\title{Maxwell Equations  in PTC's Sector Bend}
\author{E. Forest}
%\date{} % Activate to display a given date or no date (if empty),
         % otherwise the current date is printed 

\begin{document}
\maketitle

\section{Useful formulae}
%
%
%]|Expr|[#b @`b___})2# b'4" Chicago^: ;bP8&c0!*,^"!Symbol^:!&c0  /1|
%|.T""*|:"&c0!*u"#Helvetica|:#,F:",]:#,F:! <c!$1(&$^&c0  .V^:"&c0!*y|
%|_$^u^:!&c0  f_:"&c0!*,M<2^1^r}$^:!&c0  .V^f_$^:"&c0!*u^y_}}<c$5Q|
%|^r},K<2^1^r}<c!$1(&$^:!&c0  .V^f_$^:"&c0!*u^r_,M$^:!&c0  .V^:"&c0!*r|
%|_r$^u^:!&c0  f_}}<c$5Q^:"&c0!*y},K<c!$1(%$^:!&c0  .V^:"&c0!*r|
%|_$^u^y_,M$^:!&c0  .V^:"&c0!*y_$^u^r_}}<c$5Q^:!&c0  f}: &c0!*,\|
%|}& b!( b"0 b#8 b$@ b%H b&P!WW}]|[
\begin{eqnarray}\nabla \times u&=
& \left({{\partial }_{y}{u}_{\phi }-{1 \over r}{\partial }_{\phi }{u}_{y}}\right)\widehat{r}+{1 \over r}\left({{\partial }_{\phi }{u}_{r}-{\partial }_{r}r{u}_{\phi }}\right)\widehat{y}+\left({{\partial }_{r}{u}_{y}-{\partial }_{y}{u}_{r}}\right)\widehat{\phi }\end{eqnarray}
%
%
%]|Expr|[#b @`b___}):# b'4" Chicago^: ;bP8&c0!*,^"!Symbol^:!&c0  /1|
%|""*|:"&c0!*f:! "#Helvetica|:#,F:",]:#,F:! $^&c0  .V^:"&c0!*r_|
%|f <c$5Q^r} ,K $^:!&c0  .V^:"&c0!*y_f <c$5Q^y} ,K <2^1^r}$^:!&c0  .V|
%|^f_:"&c0!*f: ,\}& b!( b"0 b#8 b$@ b%H b&P!WW}]|[
\begin{eqnarray}\nabla f &=
& {\partial }_{r}f\ \widehat{r}\ +\ {\partial }_{y}f\ \widehat{y}\ +\ {1 \over r}{\partial }_{\phi }f\end{eqnarray}
%
%
%]|Expr|[#b @`b___})6# b'4" Chicago^: ;bP8&c0!*,^"!Symbol^:!&c0  /1|
%|/7""*|:"&c0!*u"#Helvetica|:#,F:",]:#,F:! <2^:"1^r}$^:!&c0  .V|
%|^:"&c0!*r_r$^u^r_ ,K$^:!&c0  .V^:"&c0!*y_$^u^y_,K<2^1^r}$^:!&c0  .V|
%|^f_$^:"&c0!*u^:!&c0  f_:"&c0!*: ,\}& b!( b"0 b#8 b$@ b%H b&P!WW}]|[
\begin{eqnarray}\nabla \cdot u&=
& {1 \over r}{\partial }_{r}r{u}_{r}\ +{\partial }_{y}{u}_{y}+{1 \over r}{\partial }_{\phi }{u}_{\phi }\end{eqnarray}
%

If we use the more convenient $(x,y,s)$ variable defined as
%
%]|Expr|[#b @`b___})=# b'4" Chicago^: ;bP8&c0!*,^"!*|:!h""Symbol^:" |
%|"#Helvetica|:#,F:!,]:#,F:" <2^:!1^$^r^0_}: ;bP;/":#;bP8 1,Khx|
%|:" :#,F:!,]:#,F:" :!hr: ;bP;/":#;bP8 hs:" :#,F:!,]:#,F:" &c0  f|
%|: &c0!*,\}& b!( b"0 b#8 b$@ b%H b&P!WW}]|[
\begin{eqnarray}h &=
& {1 \over {r}_{0}}\nonumber \\
 1+hx &=
& hr\nonumber \\
 hs &=
& \phi \end{eqnarray}
%
Then the curl, gradient and divergence  can be rewritten:
%
%]|Expr|[#b @`b___})9# b'4" Chicago^: ;bP8&c0!*,^"!Symbol^:!&c0  /1|
%|.T""*|:"&c0!*u "#Helvetica|:#,F:",]:#,F:! <c!$1(&$^&c0  .V^:"&c0!*y|
%|_$^u^s_,M<2^1(#1,Khx}}$^:!&c0  .V^:"&c0!*s_$^u^y_}}<c$5Q^x}: ;bP;/"|
%|:#;bP8 ,F,F:",K<2^1(#1,Khx}}<c!$1(/$^:!&c0  .V^:"&c0!*s_$^u^x|
%|_,M:#-<vphantom-[<2^:"1(#1,Khx}}:#-]$^:!&c0  .V^:"&c0!*x_,H1,K|
%|hx,I$^u^s_}}<c$5Q^y}: ;bP;/":#;bP8 ,F,F:",K<c!$1(*$^:!&c0  .V|
%|^:"&c0!*x_$^u^y_,M:#-<vphantom-[<2^:"1(#1,Khx}}:#-]$^:!&c0  .V|
%|^:"&c0!*y_$^u^x_}}<c$5Q^s}: ,\}& b!( b"0 b#8 b$@ b%H b&P!WW}]|[
\begin{eqnarray}\nabla \times u\ &=
& \left({{\partial }_{y}{u}_{s}-{1 \over 1+hx}{\partial }_{s}{u}_{y}}\right)\widehat{x}\nonumber \\
 &&+{1 \over 1+hx}\left({{\partial }_{s}{u}_{x}-\vphantom{{1 \over 1+hx}}{\partial }_{x}(1+hx){u}_{s}}\right)\widehat{y}\nonumber \\
 &&+\left({{\partial }_{x}{u}_{y}-\vphantom{{1 \over 1+hx}}{\partial }_{y}{u}_{x}}\right)\widehat{s}\end{eqnarray}
%
%
%]|Expr|[#b @`b___}):# b'4" Chicago^: ;bP8&c0!*,^"!Symbol^:!&c0  /1|
%|""*|:"&c0!*f:! "#Helvetica|:#,F:",]:#,F:! $^&c0  .V^:"&c0!*x_|
%|f <c$5Q^x} ,K $^:!&c0  .V^:"&c0!*y_f <c$5Q^y} ,K <2^1(#1,Khx}}|
%|$^:!&c0  .V^:"&c0!*s_f: ,\}& b!( b"0 b#8 b$@ b%H b&P!WW}]|[
\begin{eqnarray}\nabla f &=
& {\partial }_{x}f\ \widehat{x}\ +\ {\partial }_{y}f\ \widehat{y}\ +\ {1 \over 1+hx}{\partial }_{s}f\end{eqnarray}
%
%
%]|Expr|[#b @`b___})6# b'4" Chicago^: ;bP8&c0!*,^"!Symbol^:!&c0  /1|
%|/7""*|:"&c0!*u"#Helvetica|:#,F:",]:#,F:! <2^:"1(#1,Khx}}$^:!&c0  .V|
%|^:"&c0!*x_<c!$1(#1,Khx}}$^u^r_ ,K$^:!&c0  .V^:"&c0!*y_$^u^y_,K|
%|<2^1(#1,Khx}}$^:!&c0  .V^:"&c0!*s_$^u^s_: ,\|
%|}& b!( b"0 b#8 b$@ b%H b&P!WW}]|[
\begin{eqnarray}\nabla \cdot u&=
& {1 \over 1+hx}{\partial }_{x}\left({1+hx}\right){u}_{r}\ +{\partial }_{y}{u}_{y}+{1 \over 1+hx}{\partial }_{s}{u}_{s}\end{eqnarray}
%







\section{Magnetic field with vector potential: old way of PTC}

We will say that a bend has sector geometry if magnetic field and/or the electric field is invariant as we travel on a circle of radius $r$ down the center of the
element. The symmetry would be perfect if the entire element was a cyclotron; so here we neglect fringe fields at both ends.

In the case of a magnetic field, the vector potential $a_s$ must obey the equation:
%
%
%]|Expr|[#b @`b___}):# b'4" Chicago^: ;bP8&c0!*,^<c!(A($$^"!Symbol^:!&c0  /1|
%|^->^""*|:"&c0!*2,M<2^h(#1,Khx}}<2(&:!&c0  .V"#Helvetica|:#&c0!*-<|
%|hphantom-[:"x:#-]}(":!&c0  .V:"&c0!*x}}}}F:#,F,]0"$Monaco^:$ |
%| where  :#F,]<c!$1(#:"r,Kx}}$^:#a^s_:$ and "%Times|:%h,]<2^1^|
%|r}: ;8/<:";bP8eq,Zmaxb: ;8/=;bP8,\}& b!( b"0 b#8 b$@ b%H b&P!WW}]|[
\begin{eqnarray}\left\{{{\nabla }_{\perp }^{2}-{h \over 1+hx}{\partial \hphantom{x} \over \partial x}}\right\}F&=0~~{\rm w}{\rm h}{\rm e}{\rm r}{\rm e}~~F=\left({r+x}\right){a}_{s}~{\rm a}{\rm n}{\rm d} ~h=
{1 \over r}\label{eq:maxb}\end{eqnarray}
%
%
%
%]|Expr|[#b @`b___})## b'4" Helvetica|: ;bP8&c0!*,D$^"!Symbol^:!&c0  /1|
%|^->^""*|:"&c0!*2: ,D}& b!( b"0 b#8 b$@ b%H b&P!WW}]|[
${\nabla }_{\perp }^{2}$
is the transverse Laplacian %
%]|Expr|[#b @`b___})%# b'4" Helvetica|: ;bP8&c0!*,D<2(&$^"!Symbol^:!&c0  .V|
%|_^""*|:"&c0!*2: -<hphantom-[:"x: -]}^$(":!&c0  .V:"&c0!*x}_^2}|
%|,K<2(&$^:!&c0  .V_^:"&c0!*2: -<hphantom-[:"x: -]}^$(":!&c0  .V|
%|:"&c0!*y}_^2}: ,D}& b!( b"0 b#8 b$@ b%H b&P!WW}]|[
${{\partial }^{2}\hphantom{x} \over {\partial x}^{2}}+{{\partial }^{2}\hphantom{x} \over {\partial y}^{2}}$.

In a straight element, where %
%]|Expr|[#b @`b___})%# b'4" Helvetica|: ;bP8&c0!*,D"!*|:!r,]""Symbol^:"&c0  .E|
%|: &c0!*,D}& b!( b"0 b#8 b$@ b%H b&P!WW}]|[
$r=
\infty $,
the solutions if \eq{eq:maxb} used in PTC, are just:
%
%]|Expr|[#b @`b___})3# b'4" Chicago^: ;bP8&c0!*,^$^"!*|:!F^n(#|
%|h,]0}""Helvetica|:",F,],M<2^$^b^n_^n}Re<c!(A^$^<c!$1(#x,Kiy}}|
%|_^n},K<2^$^a^n_^n}Im<c!(A^$^<c!$1(#x,Kiy}}_^n}  : ;8/<:!;bP8eq|
%|,Zhsol: ;8/=:";bP8 : ,\}& b!( b"0 b#8 b$@ b%H b&P!WW}]|[
\begin{eqnarray}{F}_{n}^{h=
0}&=-{{b}_{n} \over n}Re\left\{{{\left({x+iy}\right)}^{n}}\right\}+{{a}_{n} \over n}Im\left\{{{\left({x+iy}\right)}^{n}}\right\}  \label{eq:hsol} \end{eqnarray}
%
%
These solutions have the property that, in the midplane $y=0$,  the magnetic field is given by:
%
%]|Expr|[#b @`b___})7# b'4" Chicago^: ;bP8&c0!*,^$^"!*|:!b^y_,]|
%|<c%#D("$^b^n_$^x_(#n,M1}}("""Helvetica|:"-<sum}^:!n_}"#Monaco^:# |
%|  and   $^:!b^x_,]<c%#D("$^a^n_$^x_(#n,M1}}(":"-<sum}^:!n_}:" |
%|: ;8/<:!;bP8eq,Zst,Zanbn: ;8/=;bP8,\}& b!( b"0 b#8 b$@ b%H b&P!WW}]|[
\begin{eqnarray}{b}_{y}=
\sum\limits_{n}^{} {b}_{n}{x}^{n-1}~~~{\rm a}{\rm n}{\rm d} ~~~{b}_{x}=
\sum\limits_{n}^{} {a}_{n}{x}^{n-1} \label{eq:st:anbn}\end{eqnarray}
%
%
Of course the full solution is given by an infinite sum:
%
%]|Expr|[#b @`b___}),# b'4" Chicago^: ;bP8&c0!*,^$^"!*|:!F^n^h|
%|""Symbol^:" "#Helvetica|:#,F:!,]:#,F:" <c%#D)!# b'4$^&c0  D^:!&c0!*n|
%|^k}& b!( b"0 b#8 b$@ b%H b&P!WW}(":#-<sum}(%:!k,]0,L:"&c0  .E}|
%|_}: ;8&c0!*/<:!;bP8fnk: ;8/=;bP8,\}& b!( b"0 b#8 b$@ b%H b&P!WW}]|[
\begin{eqnarray}{F}_{n}^{h} &=
& \sum\limits_{k=
0,\infty }^{} {\Delta }_{n}^{k}\label{fnk}\end{eqnarray}
%
%
The idea is to set up a recursive relation to solve for the $F_n^k$ while continuously respecting \eqs{eq:maxb} and \eqe{eq:st:anbn}. Let us see how that works at order $k$. 
%
%]|Expr|[#b @`b___})5# b'4" Chicago^: ;bP8&c0!*,^$^"!Symbol^:!&c0  D|
%|^""*|:"&c0!*n(#k,K1}:! "#Helvetica|:#,F:",]:#,F$^:!&c0  /1^->|
%|(":"&c0!*,M2}<c%#D)## b'4<2^h(#1,Khx}}<2(&:!&c0  .V:#&c0!*-<hphantom|
%|-[:"x:#-]}(":!&c0  .V:"&c0!*x}}$^:!&c0  D^:"&c0!*n^:!&c0  a|
%|}& b!( b"0 b#8 b$@ b%H b&P!WW}(":#&c0!*-<sum}(%:!&c0  a:"&c0!*,]|
%|0,Lk}_},K$^C(#k,K1}_$^F(#n,Kk}^0  : ;8/<:";bP8eq,Zsol,Zanbn: ;8/=|
%|;bP8,\}& b!( b"0 b#8 b$@ b%H b&P!WW}]|[
\begin{eqnarray}{\Delta }_{n}^{k+1} &=
&{\nabla }_{\perp }^{-2}\sum\limits_{\alpha =
0,k}^{} {h \over 1+hx}{\partial \hphantom{x} \over \partial x}{\Delta }_{n}^{\alpha }+{C}_{k+1}{F}_{n+k}^{0}\ \ \label{eq:sol:anbn}\end{eqnarray}
%
%
To solve this equation, one first initializes it at $k=0$ with 
%
%]|Expr|[#b @`b___})## b'4" Helvetica|: ;bP8&c0!*,D$^"!*|:!F^n|
%|(#h,]0}: ,D}& b!( b"0 b#8 b$@ b%H b&P!WW}]|[
${F}_{n}^{h=
0}$
of \eq{eq:hsol}. The operator %
${\nabla }_{\perp }^{-2}$ of \eq{eq:sol:anbn} find a solution for the inverse Laplacian. In phasors variable 
%
%]|Expr|[#b @`b___})%# b'4" Helvetica|: ;bP8&c0!*,D<c$%!^"!*|:!u}|
%|,]<c!$1('x,Kiy,Lx,Miy}}: ,D}& b!( b"0 b#8 b$@ b%H b&P!WW}]|[
$\vec{u}=
\left({x+iy,x-iy}\right)$
the inverse Laplacian is   given by 
%
%]|Expr|[#b @`b___})># b'4" Chicago^: ;bP8&c0!*,^$^"!Symbol^:!&c0  /1|
%|^->("""*|:"&c0!*,M2}f,H<c$%!^u},I"#Helvetica|:#,F,]<2^1^4}<c" #|
%|^<c" #(&f du d<c$%A^u}}_^<c$%A^u}}_^u} ,K C $^f_^0"$Monaco^:$ |
%| where  $^:!&c0  /1^->^:"&c0!*2$^:#f_^0,]0: ;8/<:#eq,Z;bP8inv|
%|: ;8/=;bP8,\}& b!( b"0 b#8 b$@ b%H b&P!WW}]|[
\begin{eqnarray}{\nabla }_{\perp }^{-2}f(\vec{u})&={1 \over 4}\int_{}^{u}\int_{}^{\overline{u}}f du d\overline{u} + C {f}^{0}~~{\rm w}{\rm h}{\rm e}{\rm r}{\rm e}~~{\nabla }_{\perp }^{2}{f}^{0}=0\label{eq:inv}\end{eqnarray}
%

in \eq{eq:inv} the first term is an antiderivative with with respect to $u$ and $ \overline{u}$. The additive term is an harmonic solution of the type of \eq{eq:hsol}. 

To obtain the solution to the order of truncation one iterates \eq{eq:sol:anbn} starting with the harmonic solution %
%]|Expr|[#b @`b___})%# b'4" Helvetica|: ;bP8&c0!*,D$^"!Symbol^:!&c0  D|
%|^""*|:"&c0!*n(#k,]0},]$^F^n^0: ,D}& b!( b"0 b#8 b$@ b%H b&P!WW}]|[
${\Delta }_{n}^{k=
0}=
{F}_{n}^{0}$.

The only issue is how does one set the constant ${C}_{k+1}$? One notices that the harmonic solution  ${F}_{n}^{0}$ already obeys \eq{eq:st:anbn}. Therefore it suffices to  choose ${C}_{k+1}$ so as to remove leading order terms of the form $C x^{n+k+1}$  in ${\Delta }_{n}^{k+1}$ by adding a term proportional to ${F}_{n}^{k+1}$. Higher order terms which violate \eq{eq:st:anbn} while be handled at the next iterations.



\section{Magnetic field with scalar potential: new way of PTC}

\vskip 0.5cm
{\tt \large \bf \it See routine get_bend_magnetic_potential in Se_status.f90}
\vskip 0.5cm

We will say that a bend has sector geometry if magnetic field and/or the electric field is invariant as we travel on a circle of radius $r$ down the center of the
element. The symmetry would be perfect if the entire element was a cyclotron; so here we neglect fringe fields at both ends.

In the case of a magnetic field, the scalar potential $V$ must obey the equation:
%
%]|Expr|[#b @`b___}),# b'4" Chicago^: ;bP8&c0!*,^<c!(A($$^"!Symbol^:!&c0  /1|
%|^->^""*|:"&c0!*2,K<2^h(#1,Khx}}<2(&:!&c0  .V"#Helvetica|:#&c0!*-<|
%|hphantom-[:"x:#-]}(":!&c0  .V:"&c0!*x}}}}V:#,F,]0"$Monaco^:$ |
%|: ;8/<:";bP8eq,Zmaxbv: ;8/=;bP8,\}& b!( b"0 b#8 b$@ b%H b&P!WW}]|[
\begin{eqnarray}\left\{{{\nabla }_{\perp }^{2}+{h \over 1+hx}{\partial \hphantom{x} \over \partial x}}\right\}V&=0~\label{eq:maxbv}\end{eqnarray}
%
%
%
%
%]|Expr|[#b @`b___})## b'4" Helvetica|: ;bP8&c0!*,D$^"!Symbol^:!&c0  /1|
%|^->^""*|:"&c0!*2: ,D}& b!( b"0 b#8 b$@ b%H b&P!WW}]|[
${\nabla }_{\perp }^{2}$
is the transverse Laplacian %
%]|Expr|[#b @`b___})%# b'4" Helvetica|: ;bP8&c0!*,D<2(&$^"!Symbol^:!&c0  .V|
%|_^""*|:"&c0!*2: -<hphantom-[:"x: -]}^$(":!&c0  .V:"&c0!*x}_^2}|
%|,K<2(&$^:!&c0  .V_^:"&c0!*2: -<hphantom-[:"x: -]}^$(":!&c0  .V|
%|:"&c0!*y}_^2}: ,D}& b!( b"0 b#8 b$@ b%H b&P!WW}]|[
${{\partial }^{2}\hphantom{x} \over {\partial x}^{2}}+{{\partial }^{2}\hphantom{x} \over {\partial y}^{2}}$.

In a straight element, where %
%]|Expr|[#b @`b___})%# b'4" Helvetica|: ;bP8&c0!*,D"!*|:!r,]""Symbol^:"&c0  .E|
%|: &c0!*,D}& b!( b"0 b#8 b$@ b%H b&P!WW}]|[
$r=
\infty $,
the solutions if \eq{eq:maxbv} used in PTC, are just:
%
%]|Expr|[#b @`b___})3# b'4" Chicago^: ;bP8&c0!*,^$^"!*|:!F^n(#|
%|h,]0}""Helvetica|:",F,]<2^$^b^n_^n}Re<c!(A^$("i<c!$1(#x,Kiy}}}|
%|_^n},M<2^$^a^n_^n}Re<c!(A^$^<c!$1(#x,Kiy}}_^n}  : ;8/<:!;bP8eq|
%|,Zhsolv: ;8/=:";bP8 : ,\}& b!( b"0 b#8 b$@ b%H b&P!WW}]|[
\begin{eqnarray}{F}_{n}^{h=
0}&={{b}_{n} \over n}Re\left\{{{i\left({x+iy}\right)}^{n}}\right\}-{{a}_{n} \over n}Re\left\{{{\left({x+iy}\right)}^{n}}\right\}  \label{eq:hsolv} \end{eqnarray}
%
%
These solutions have the property that, in the midplane $y=0$,  the magnetic field is given by:
%
%]|Expr|[#b @`b___})7# b'4" Chicago^: ;bP8&c0!*,^$^"!*|:!b^y_,]|
%|<c%#D("$^b^n_$^x_(#n,M1}}("""Helvetica|:"-<sum}^:!n_}"#Monaco^:# |
%|  and   $^:!b^x_,]<c%#D("$^a^n_$^x_(#n,M1}}(":"-<sum}^:!n_}:" |
%|: ;8/<:!;bP8eq,Zst,Zanbn: ;8/=;bP8,\}& b!( b"0 b#8 b$@ b%H b&P!WW}]|[
\begin{eqnarray}{b}_{y}=
\sum\limits_{n}^{} {b}_{n}{x}^{n-1}~~~{\rm a}{\rm n}{\rm d} ~~~{b}_{x}=
\sum\limits_{n}^{} {a}_{n}{x}^{n-1} \label{eq:st:anbnv}\end{eqnarray}
%
%
Of course the full solution is given by an infinite sum:
%
%]|Expr|[#b @`b___}),# b'4" Chicago^: ;bP8&c0!*,^$^"!*|:!F^n^h|
%|""Symbol^:" "#Helvetica|:#,F:!,]:#,F:" <c%#D)!# b'4$^&c0  D^:!&c0!*n|
%|^k}& b!( b"0 b#8 b$@ b%H b&P!WW}(":#-<sum}(%:!k,]0,L:"&c0  .E}|
%|_}: ;8&c0!*/<:!;bP8fnk: ;8/=;bP8,\}& b!( b"0 b#8 b$@ b%H b&P!WW}]|[
\begin{eqnarray}{F}_{n}^{h} &=
& \sum\limits_{k=
0,\infty }^{} {\Delta }_{n}^{k}\label{fnkv}\end{eqnarray}
%
%
The idea is to set up a recursive relation to solve for the $F_n^k$ while continuously respecting \eqs{eq:maxbv} and \eqe{eq:st:anbnv}. Let us see how that works at order $k$. 
%
%]|Expr|[#b @`b___})5# b'4" Chicago^: ;bP8&c0!*,^$^"!Symbol^:!&c0  D|
%|^""*|:"&c0!*n(#k,K1}:! "#Helvetica|:#,F:",]:#,F,M$^:!&c0  /1^|
%|->(":"&c0!*,M2}<c%#D)## b'4<2^h(#1,Khx}}<2(&:!&c0  .V:#&c0!*-<|
%|hphantom-[:"x:#-]}(":!&c0  .V:"&c0!*x}}$^:!&c0  D^:"&c0!*n^:!&c0  a|
%|}& b!( b"0 b#8 b$@ b%H b&P!WW}(":#&c0!*-<sum}(%:!&c0  a:"&c0!*,]|
%|0,Lk}_},K$^C(#k,K1}_$^F(#n,Kk}^0  : ;8/<:";bP8eq,Zsol,Zanbnv: ;8/=|
%|;bP8,\}& b!( b"0 b#8 b$@ b%H b&P!WW}]|[
\begin{eqnarray}{\Delta }_{n}^{k+1} &=
&-{\nabla }_{\perp }^{-2}\sum\limits_{\alpha =
0,k}^{} {h \over 1+hx}{\partial \hphantom{x} \over \partial x}{\Delta }_{n}^{\alpha }+{C}_{k+1}{F}_{n+k}^{0}\ \ \label{eq:sol:anbnv}\end{eqnarray}
%
%
To solve this equation, one first initializes it at $k=0$ with 
%
%]|Expr|[#b @`b___})## b'4" Helvetica|: ;bP8&c0!*,D$^"!*|:!F^n|
%|(#h,]0}: ,D}& b!( b"0 b#8 b$@ b%H b&P!WW}]|[
${F}_{n}^{h=
0}$
of \eq{eq:hsolv}. The operator %
${\nabla }_{\perp }^{-2}$ of \eq{eq:sol:anbnv} find a solution for the inverse Laplacian. In phasors variable 
%
%]|Expr|[#b @`b___})%# b'4" Helvetica|: ;bP8&c0!*,D<c$%!^"!*|:!u}|
%|,]<c!$1('x,Kiy,Lx,Miy}}: ,D}& b!( b"0 b#8 b$@ b%H b&P!WW}]|[
$\vec{u}=
\left({x+iy,x-iy}\right)$
the inverse Laplacian is   given by 
%
%]|Expr|[#b @`b___})># b'4" Chicago^: ;bP8&c0!*,^$^"!Symbol^:!&c0  /1|
%|^->("""*|:"&c0!*,M2}f,H<c$%!^u},I"#Helvetica|:#,F,]<2^1^4}<c" #|
%|^<c" #(&f du d<c$%A^u}}_^<c$%A^u}}_^u} ,K C $^f_^0"$Monaco^:$ |
%| where  $^:!&c0  /1^->^:"&c0!*2$^:#f_^0,]0: ;8/<:#eq,Z;bP8invv|
%|: ;8/=;bP8,\}& b!( b"0 b#8 b$@ b%H b&P!WW}]|[
\begin{eqnarray}{\nabla }_{\perp }^{-2}f(\vec{u})&={1 \over 4}\int_{}^{u}\int_{}^{\overline{u}}f du d\overline{u} + C {f}^{0}~~{\rm w}{\rm h}{\rm e}{\rm r}{\rm e}~~{\nabla }_{\perp }^{2}{f}^{0}=0\label{eq:invv}\end{eqnarray}
%

%

in \eq{eq:invv} the first term is an antiderivative with with respect to $u$ and $ \overline{u}$. The additive term is an harmonic solution of the type of \eq{eq:hsolv}. 

To obtain the solution to the order of truncation one iterates \eq{eq:sol:anbnv} starting with the harmonic solution %
%]|Expr|[#b @`b___})%# b'4" Helvetica|: ;bP8&c0!*,D$^"!Symbol^:!&c0  D|
%|^""*|:"&c0!*n(#k,]0},]$^F^n^0: ,D}& b!( b"0 b#8 b$@ b%H b&P!WW}]|[
${\Delta }_{n}^{k=
0}=
{F}_{n}^{0}$.

The only issue is how does one set the constant ${C}_{k+1}$? One notices that the harmonic solution  ${F}_{n}^{0}$ already obeys \eq{eq:st:anbnv}. Therefore it suffices to  choose ${C}_{k+1}$ so as to remove leading order terms of the form $C x^{n+k+1}$  in ${\Delta }_{n}^{k+1}$ by adding a term proportional to ${F}_{n}^{k+1}$. Higher order terms which violate \eq{eq:st:anbnv} while be handled at the next iterations.


\section{Electric Field}

\vskip 0.5cm
{\tt \large \bf \it See routine    get_bend_electric_coeff in Se_status.f90}
\vskip 0.5cm

The electric field should also obey Maxwell's equation. If we represent it by a potential $F$, its Laplacian must vanish. In cylindrical coordinates, it is given by the equation:
%
%]|Expr|[#b @`b___}):# b'4" Chicago^: ;bP8&c0!*,^$^"!Symbol^:!&c0  /1|
%|^->^""*|:"&c0!*2F:! "#Helvetica|:#,F:",]:#,F:! <c!(A)&# b'4<2|
%|^:"1(#1,Khx}}<2(&:!&c0  .V:#&c0!*-<hphantom-[:"x:#-]}(":!&c0  .V|
%|:"&c0!*x}}<c!$1(#1,Khx}}<2(&:!&c0  .V:#&c0!*-<hphantom-[:"x:#-]}|
%|(":!&c0  .V:"&c0!*x}},K<2(&$^:!&c0  .V_^:"&c0!*2:#-<hphantom-[|
%|:"x:#-]}^$(":!&c0  .V:"&c0!*y}_^2}}& b!( b"0 b#8 b$@ b%H b&P!WW}}|
%|F: ;bP;/":#;bP8 :! :#,F:",]:#,F:! <c!(A($$^&c0  /1^->^:"&c0!*2|
%|,K<2^h(#1,Khx}}<2(&:!&c0  .V:#&c0!*-<hphantom-[:"x:#-]}(":!&c0  .V|
%|:"&c0!*x}}}}F : ;8/<:#;bP8eq,Zelec: ;8/=;bP8,\|
%|}& b!( b"0 b#8 b$@ b%H b&P!WW}]|[
\begin{eqnarray}{\nabla }_{\perp }^{2}F &=
& \left\{{{1 \over 1+hx}{\partial \hphantom{x} \over \partial x}\left({1+hx}\right){\partial \hphantom{x} \over \partial x}+{{\partial }^{2}\hphantom{x} \over {\partial y}^{2}}}\right\}F\nonumber \\
  &=
& \left\{{{\nabla }_{\perp }^{2}+{h \over 1+hx}{\partial \hphantom{x} \over \partial x}}\right\}F\ \label{eq:elec}\end{eqnarray}
%
The zeroth order solution is:
%
%]|Expr|[#b @`b___})3# b'4" Chicago^: ;bP8&c0!*,^$^"!*|:!F^n(#|
%|h,]0}""Helvetica|:",F,],M<2^$^b^n_^n}Re<c!(A^$^<c!$1(#x,Kiy}}|
%|_^n},M<2^$^a^n_^n}Im<c!(A^$^<c!$1(#x,Kiy}}_^n}  : ;8/<:!;bP8eq|
%|,Zhsolev: ;8/=:";bP8 : ,\}& b!( b"0 b#8 b$@ b%H b&P!WW}]|[
\begin{eqnarray}{F}_{n}^{h=
0}&=-{{b}_{n} \over n}Re\left\{{{\left({x+iy}\right)}^{n}}\right\}-{{a}_{n} \over n}Im\left\{{{\left({x+iy}\right)}^{n}}\right\}  \label{eq:hsolev} \end{eqnarray}
%
or equivalently
%
%]|Expr|[#b @`b___})3# b'4" Chicago^: ;bP8&c0!*,^$^"!*|:!F^n(#|
%|h,]0}""Helvetica|:",F,],M<2^$^b^n_^n}Re<c!(A^$^<c!$1(#x,Kiy}}|
%|_^n},K<2^$^a^n_^n}Re<c!(A^$("i<c!$1(#x,Kiy}}}_^n}  : ;8/<:!;bP8eq|
%|,Zhsolevi: ;8/=:";bP8 : ,\}& b!( b"0 b#8 b$@ b%H b&P!WW}]|[
\begin{eqnarray}{F}_{n}^{h=
0}&=-{{b}_{n} \over n}Re\left\{{{\left({x+iy}\right)}^{n}}\right\}+{{a}_{n} \over n}Re\left\{{{i\left({x+iy}\right)}^{n}}\right\}  \label{eq:hsolevi} \end{eqnarray}
%

The electric field is defined as:
%
%]|Expr|[#b @`b___}),# b'4" Chicago^: ;bP8&c0!*,^<c$%!^"!*|:!E}|
%|""Symbol^:" "#Helvetica|:#,F:!,]:#,F:" :!,M<c$%!^:"&c0  /1}:!&c0!*F|
%|: ,\}& b!( b"0 b#8 b$@ b%H b&P!WW}]|[
\begin{eqnarray}\vec{E} &=
& -\vec{\nabla }F\end{eqnarray}

The electric coefficients $a_n$ and $b_n$ are determined by a relation similar to \eq{eq:st:anbnv}:
%
%]|Expr|[#b @`b___}).# b'4" Chicago^: ;bP8&c0!*,^$^"!*|:!E^x_,]|
%|<c%#D("$^b^n_$^x_(#n,M1}}("""Helvetica|:"-<sum}^"#Symbol^:#&c0  a|
%|_}"$Monaco^:$&c0!*  and   $^:!E^y_,]<c%#D("$^a^n_$^x_(#n,M1}}|
%|(":"-<sum}^:#&c0  a_}: &c0!*,\}& b!( b"0 b#8 b$@ b%H b&P!WW}]|[
\begin{eqnarray}{E}_{x}=
\sum\limits_{\alpha }^{} {b}_{n}{x}^{n-1}~~{\rm a}{\rm n}{\rm d} ~~~{E}_{y}=
\sum\limits_{\alpha }^{} {a}_{n}{x}^{n-1}\end{eqnarray}
%
%

PTC has an electric septum. This is a straight dipole element producing a  vertical electric field. Remarkably, the body of that bend can be exactly solved.
Of course a zero curvature electric sector bend should produce the same result. This is something we ought to check and document.

The following piece of code checks most assertions of this document. Notice that PTC has two new parameters {\tt volt_c=1.e-3} and {\tt volt_i=1}. These are used to enforce megavolts in PTC. In reality the code uses gigavolts. Here by setting {\tt volt_c} to 1, I use gigavolts. {\tt volt_i}   is an input parameter. We can enforce megavolts in the code computations (deep in) or at the input level when we create the element. Anyway in this example, everything  is in gigavolts  {\tt volt_c=1} .

\begin{verbatim}
call GET_ONE(p0c=p0c)
volt_c=1
voltage=1.d-6*p0c
solve_electric=my_true
B  = SBEND("B", L=2.54948D0,ANGLE=0.d0) 
CALL ADD(B,-1,1,voltage,electric=my_true)

QF =ELSEPARATOR("BB", L=2.54948D0,E=voltage)
!QF%MAG%phas=pi/2d0 ;QF%MAGp%phas=QF%MAG%phas;

X=0.001D0
call init(8,2)
call alloc(y);call alloc(eb);call alloc(phi);call alloc(bf)
y(1)=morph(1.0d0.mono.1);
y(3)=morph(1.0d0.mono.2);
call GETELECTRIC(B%mag%tp10,BFr,phir,EBR,x)

WRITE(6,*) BFR
WRITE(6,*) phir
CALL GETELECTRIC(QF%MAG%SEP15,EBR,phir,X)
WRITE(6,*) EBR
WRITE(6,*) phir
call GETELECTRIC(B%magp%tp10,BF,phi,EB,y)
do i=1,3
call print(bf(i),6)
enddo
call print(phi,6)
b%dir=1
x=0.001d0 
call track(b,x,state)
write(6,*)x
x=0.001d0 
call track(QF,x,state)
write(6,*)x
stop
\end{verbatim}


The result is:


\begin{verbatim}
  0.000000000000000E+000  1.000000000000000E-006  0.000000000000000E+000
 -9.999999999999999E-010
  0.000000000000000E+000  1.000000000000000E-006  0.000000000000000E+000
 -1.000000000000000E-009

 etall    1, NO =    8, NV =    2, INA =   40
 *********************************************

   ALL COMPONENTS ZERO
      NO =     8      NV =     2
    -1   0.000000000000000       0  0

 etall    1, NO =    8, NV =    2, INA =   41
 *********************************************

    I  COEFFICIENT          ORDER   EXPONENTS
      NO =     8      NV =     2
   0  0.1000000000000000E-05   0  0
    -1   0.000000000000000       0  0
  0.000000000000000E+000

 etall    1, NO =    8, NV =    2, INA =   30
 *********************************************

    I  COEFFICIENT          ORDER   EXPONENTS
      NO =     8      NV =     2
   1 -0.1000000000000000E-05   0  1
    -1   0.000000000000000       0  0
  3.546219861200883E-003  1.000000000000000E-003  3.550377891682065E-003
  1.003266041983185E-003  1.000000000000000E-003 -6.323016230930416E-004
  3.546219861200894E-003  1.000000000000000E-003  3.550377891682071E-003
  1.003266041983188E-003  1.000000000000000E-003 -6.323016230926903E-004
\end{verbatim}

So all is fine: the electric septum produces a vertical kick,i.e., an $E_y$. The results agrees with the sector bend integration. Notice that the parameter {\tt solve_electric} is set to true before creating the element. This forces non-symplectic integration of a sector bend with electric and magnetic field.

Now, I will produce an horizontal kick. For the septum, it is achieved using a rotation of $\pi /2$.

\begin{verbatim}
B  = SBEND("B", L=2.54948D0,ANGLE=0.d0)  
CALL ADD(B,1,1,voltage,electric=my_true)

QF =ELSEPARATOR("BB", L=2.54948D0,E=voltage)
QF%MAG%phas=pi/2d0 ;QF%MAGp%phas=QF%MAG%phas;
\end{verbatim}

Tracking results  agrees.

\section{Electric hard edge fringe effects}

 %
%]|Expr|[#>`b___})b B# b'4" Chicago^: ;bP8&c55*,^$^"!*~:!a^x_""Helvetica~:",F|
%|:!,]:",F:!/0<2(!yU}(!Brho}}: ;bP;/":";bP8 $^:!a^y_:",F:!,]:",F|
%|<2(!:!xU}(!Brho}}: ;bP;/":";bP8 :!U:",F:!,]:",F<c%#D^<2(&:!,H|
%|/01$^,I_^n$^r_("2n}$^B^z^<c!=Q("2n}}}(%$^2_($2n,K1}n,A,Hn,K1,I,A}}|
%|(":"-<sum}(#:!n,]0}^"#Symbol^:#.E}: ;bP;/":";bP8 $^:!B^z^<c!=Q|
%|("2n}}:",F:!,]:",F<2^$^d_("2n}("d$^z_("2n}}}$^:!B^z_,H0,L0,Lz|
%|,I,N: .O,\}& b!( b"0 b#8 b$@ b%H b&P!WW}]|[
\begin{eqnarray}{a}_{x}&=
&-{yU \over Brho}\nonumber \\
 {a}_{y}&=
&{xU \over Brho}\nonumber \\
 U&=
&\sum\limits_{n=
0}^{\infty } {(-1{)}^{n}{r}^{2n}{B}_{z}^{\left[{2n}\right]} \over {2}^{2n+1}n!(n+1)!}\nonumber \\
 {B}_{z}^{\left[{2n}\right]}&=
&{{d}^{2n} \over d{z}^{2n}}{B}_{z}(0,0,z).\nonumber 
\end{eqnarray}



The scaled vector potential for this field is given by 
%
%]|Expr|[#>`b___})-# b'4" Chicago^: ;bP8&c55*,^$^"!*~:!a^z_""Helvetica~:",F|
%|:!,]:",F:!/0x b,Hz,I,N: ,\}& b!( b"0 b#8 b$@ b%H b&P!WW}]|[
\begin{eqnarray}{a}_{z}&=
&-x\ b(z).\end{eqnarray}
%
It is easy to check that this vector potential does not obey Maxwell's\index{Maxwell's equations} equations---%
%]|Expr|[#>`b___})&# b'4" Helvetica~: ;bP8&c55*,D"!Symbol^:!&c0  /1.T/1.T|
%|<c!$1(+""*~:"&c55*0,L0,L/0x b,Hz,I}}:!&c0  .Y:"&c55*0: ,D}& b!( b"0 b#8 b$@ b%H b&P!WW}]|[
$\nabla \times \nabla \times \left({0,0,-x\ b(z)}\right)\ne 0$.
However we can introduce an $x$-component to the vector potential of the form
%
%]|Expr|[#>`b___})(# b'4" Chicago^: ;bP8&c55*,^$^"!*~:!a^x_""Helvetica~:",F|
%|:!,]:",F<c%#D(&$^a^x^n:!,Hz,I $^y_("2n}}(":"-<sum}(#:!n,]1}^"#Symbol^:#.E}|
%|:!,L: ,\}& b!( b"0 b#8 b$@ b%H b&P!WW}]|[
\begin{eqnarray}{a}_{x}&=
&\sum\limits_{n=
1}^{\infty } {a}_{x}^{n}(z)\ {y}^{2n},\end{eqnarray}
%
and require that 
%
%]|Expr|[#>`b___})## b'4" Helvetica~: ;bP8&c55*,D<c!$1(#$^"!*~:!a|
%|^x_,L0,L$^a^z_}}: ,D}& b!( b"0 b#8 b$@ b%H b&P!WW}]|[
$\left({{a}_{x},0,{a}_{z}}\right)$
be curl-free.  The solution is found  immediately to be 
%
%]|Expr|[#>`b___})+# b'4" Chicago^: ;bP8&c55*,^$^"!*~:!a^x_""Helvetica~:",F|
%|:!,]:",F<c%#D("<2("$(!:!,H/01,I}_^n$^b^z^<c!=Q($2n/01}}}(%,H2|
%|n,I,A}}$^y_("2n}}(":"-<sum}(#:!n,]1}^"#Symbol^:#.E}:!,N: ;8/<|
%|:!;bP8bendax: ;8/=;bP8,\}& b!( b"0 b#8 b$@ b%H b&P!WW}]|[
\begin{eqnarray}{a}_{x}&=
&\sum\limits_{n=
1}^{\infty } {{(-1)}^{n}{b}_{z}^{\left[{2n-1}\right]} \over (2n)!}{y}^{2n}.\label{bendax}\end{eqnarray}
%

\section{Hamiltonian}
%
%]|Expr|[#b @`b___})=# b'4" Chicago^: ;bP8&c0!*,^"!*|:!K""Symbol^:" |
%|"#Helvetica|:#,F:!,]:#,F:" :!,M<c!$1(#1,Khx}}<b R(*1,K2<2^:"&c0  D|
%|^$^b^:!&c0!*0_},K$^:"&c0  D_^:!&c0!*2,M$^p^x^2,M$^p^y^2}_},MA|
%|  "$Monaco^:$ where :"&c0  D:!&c0!*,]:"&c0  d:!&c0!*,M:"&c0  F|
%|:!&c0!*   : ;8/<:!;bP8hamptc: ;8/=;bP8,\|
%|}& b!( b"0 b#8 b$@ b%H b&P!WW}]|[
\begin{eqnarray}K &=
& -\left({1+hx}\right)\sqrt {1+2{\Delta  \over {\beta }_{0}}+{\Delta }^{2}-{p}_{x}^{2}-{p}_{y}^{2}}-A\ \ ~{\rm w}{\rm h}{\rm e}{\rm r}{\rm e}~\Delta =
\delta -\Phi \ \ \ \label{hamptc}\end{eqnarray}
%



\end{document}

 SUBROUTINE  get_bend_coeff(s_b0t,NO1,h00,verb)
    implicit none
    integer no,n,i,k,j(2),NO1,l,mf
    type(taylor) x,y,h,df,ker,sol
    type(complextaylor) z 
    type(taylor) f,kick_x,kick_y
    type(damap) y0
    real(dp) h0,cker,cker0
     TYPE(B_CYL), intent(inout) :: s_b0t
    logical(lp),optional :: verb    
      real(dp),optional :: h00   


    if(present(verb)) call kanalnummer(mf,"internal_sol_mag.txt")
    call init(no1,1,0,0)
    no=sector_nmul

    call alloc(x,y,kick_x,kick_y)
    call alloc(z)
    call alloc(f,h,df,ker,sol)
    call alloc(y0)


      x=1.d0.mono.1
      y=1.d0.mono.2
      y0=1
      y0%v(2)=0
 !    z=x-i_*y   
      z=x+i_*y      
!!!  


    h0=1.d0
    if(present(h00)) h0=h00

    h=(1.d0+h0*x)
    do k=1,no1
    !  erect multipole
    f=dreal(-z**K/K)
    
    df=f

    do i=k,no-1 !k+1
     df=h0*(df.d.1)/h
     call  invert_laplace(df)
      sol=f+df
      sol=-(sol.d.1)/h
      j=0
      j(1)=i
      cker=(sol.sub.j)
      df=df-cker*dreal(-z**(I+1)/(I+1))
      f=f+df
       f=f.cut.(no+1)
     enddo
    if(present(verb)) then
        write(mf,*) " Erect ",k
        call clean_taylor(f,f,1.d-6)
        call print(f,mf)
         y=((f.d.1).d.1)+((f.d.2).d.2)-h0*(f.d.1)/h
        y=y.cut.(no-1)
    !    call print(y,6)
        write(mf,*) full_abs(y)
        kick_x=(f.d.1)
        kick_y=(f.d.2)
        call print(kick_x,mf)
        call print(kick_Y,mf)
       endif

       kick_x=((f.d.2)/h).cut.no   !!!  magnetic fields instead of kicks
       kick_y=-((f.d.1)/h).cut.no

!       kick_x=(f.d.1)
!       kick_y=(f.d.2)
       call clean_taylor(kick_x,kick_x,1.d-6)
       call clean_taylor(kick_y,kick_y,1.d-6)
       do l=1,s_b0t%n_mono
        j(1)=s_b0t%i(l)
        j(2)=s_b0t%j(l)
        s_b0t%b_x(k,l)=kick_x.sub.j
        s_b0t%b_y(k,l)=kick_y.sub.j
       enddo

enddo    


    do k=1,no1
    !  Skew multipole 
!   f=aimag(-z**K/K)
    f=aimag(z**K/K)
    df=f
    do i=k,no-1 !k+1
     df=h0*(df.d.1)/h
     call  invert_laplace(df)
      sol=f+df
      sol=(sol.d.2)/h
      j=0
      j(1)=i
      cker=(sol.sub.j)
!      df=df-cker*aimag(-z**(I+1)/(I+1))
      df=df-cker*aimag(z**(I+1)/(I+1))
      f=f+df
       f=f.cut.(no+1)
     enddo
    if(present(verb)) then
        write(mf,*) " Skew ",k
        call clean_taylor(f,f,1.d-6)
        call print(f,mf)
         y=((f.d.1).d.1)+((f.d.2).d.2)-h0*(f.d.1)/h
        y=y.cut.(no-1)
!        call print(y,6)
        write(mf,*) full_abs(y)
        kick_x=(f.d.1)
        kick_y=(f.d.2)
        call print(kick_x,mf)
        call print(kick_Y,mf)
       endif

       kick_x=((f.d.2)/h).cut.no   !!!  magnetic fields instead of kicks
       kick_y=-((f.d.1)/h).cut.no

!       kick_x=(f.d.1)
!       kick_y=(f.d.2)
!       kick_x=(f.d.1)
!       kick_y=(f.d.2)
       call clean_taylor(kick_x,kick_x,1.d-6)
       call clean_taylor(kick_y,kick_y,1.d-6)
       do l=1,s_b0t%n_mono
        j(1)=s_b0t%i(l)
        j(2)=s_b0t%j(l)
        s_b0t%a_x(k,l)=kick_x.sub.j
        s_b0t%a_y(k,l)=kick_y.sub.j
       enddo
enddo    

        if(present(verb)) close(mf)
    call kill(x,y,kick_x,kick_y)
    call kill(z)
    call kill(f,h,df,ker,sol)
    call kill(y0)

    end subroutine get_bend_coeff
