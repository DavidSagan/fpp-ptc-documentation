%!TEX root = ../PTC-LibUG.tex

%:copyright page
\clearpage
\thispagestyle{empty}

{\small
\noindent
The writing of this manual was supported in part by the
U.S. Department of Energy, Office of Science, Office of Nuclear Physics
under SBIR Grant No.\,DE-FG02-06ER84508.
}

%\vspace{2\baselineskip}
%\noindent
%\textsc{Notes on \PTCDocDraft\ draft}
%\begin{enumerate}
%  \item \'Etienne sent the following source files that I have not mentioned:
%  \begin{itemize}
%    \item \ptc{zzy\_run\_madx.f90} in the \ptc{ptc\_2008\_7\_21} folder,
%    \item \ptc{z\_dan\_psr\_tune\_integral.f90} in the \ptc{ptc\_2008\_8\_14} folder,
%    \item \ptc{zzz\_psr\_one\_magnet.f90} in the \ptc{ptc\_2008\_8\_14} folder.
%  \end{itemize}
%\end{enumerate}


\vfill
{\footnotesize
\setlength{\parindent}{0pt}
%\expandafter\capitalize\PTCDocDraft\ draft, release \PTCDocVersion.\\
Copyright \textcopyright\ \PTCDocYear\ Tech-X Corporation. All rights reserved.

~\par
The Polymorphic Tracking Code, \PTC, is copyright \textcopyright\ 2008
\'Etienne Forest and CERN. All rights reserved.\\
The fibre and the integration node, with their resulting linked list
types, the layout and the node layout, are based on concepts first
elaborated with J. Bengtsson. The node layout is similar to the
Lagrangian class that Bengtsson and Forest contemplated around 1990
for the \Cpp\ collaboration later known as \textsc{Classic}.

~\par
\LEGOr\ is a registered trademark of the \LEGO\ Group.\\
Windodw$^\text{\textregistered}$ is a registered trademark of
Microsoft Corporation in the United States and other countries.
All other trademarks are the property of their respective owners.

~\par
Tech-X Corporation\\
5621 Arapahoe Avenue, Suite A\\
Boulder, CO 80303

\url{http://www.txcorp.com}\\
\href{mailto:info@txcorp.com}{\nolinkurl{info@txcorp.com}}

~\par
\dmydate
Typeset \currenttime\ on \today\ using the \textsf{memoir} class in \LaTeXe.
%Typeset 15.15 on 22 October 2010 using the \textsf{memoir} class in \LaTeXe.
%Typeset using the \textsf{memoir} class in \LaTeXe.
%\ClockFrametrue
%\ClockStyle=3
%\ \clocktime
\vspace*{0.1\textheight}
}


\endinput
